
\hypertarget{refs}{}
\begin{CSLReferences}{1}{0}
\leavevmode\vadjust pre{\hypertarget{ref-0rourke_unfolding_2010}{}}%
0'Rourke, Meghan. 2010. {«~The {Unfolding}~»}. \emph{The New Yorker},
juillet.

\leavevmode\vadjust pre{\hypertarget{ref-abbott_how_1878}{}}%
Abbott, Edwin. 1878. \emph{{How to Parse: An Attempt to Apply the
Principles of Scholarship to English Grammar - With Appendixes on
Analysis, Spelling, and Punctuation}}. {Roberts Brothers}.

\leavevmode\vadjust pre{\hypertarget{ref-acland_residual_2007}{}}%
Acland, Charles R., éd. 2007. \emph{Residual {Media}}. {Minneapolis}:
{University of Minnesota Press}.

\leavevmode\vadjust pre{\hypertarget{ref-agamben_theorie_2006}{}}%
Agamben, Giorgio. 2006. {«~{Th{é}orie des dispositifs}~»}. Traduit par
Martin Rueff. \emph{Po{é}sie} N{\textdegree} 115 (1):25‑33.

\leavevmode\vadjust pre{\hypertarget{ref-aiello_communication_2022}{}}%
Aiello, Giorgia. 2022. \emph{{Communication, espace, image}}. Édité par
Marta Severo. {La petite collection ArTeC}. {Paris Dijon}: {ArTeC les
Presses du r{é}el}.

\leavevmode\vadjust pre{\hypertarget{ref-anderson_family_2010}{}}%
Anderson, Sam. 2010. {«~Family {Album}~»}. \emph{New York Magazine},
avril.

\leavevmode\vadjust pre{\hypertarget{ref-archibald_texte_2009}{}}%
Archibald, Samuel. 2009. \emph{Le {Texte} et La Technique}. Collection
Erres Essais. {Montr{é}al, (Qu{é}bec)}: {Le Quartanier}.

\leavevmode\vadjust pre{\hypertarget{ref-audet_2011}{}}%
Audet, René, et Simon Brousseau. 2011. {«~Pour Une Po{é}tique de La
Diffraction de l'{œ}uvre Litt{é}raire Num{é}rique : {L}'archive, Le
Texte et l'{œ}uvre {à} l'estompe~»}. \emph{Prot{é}e} 39 (1).
{D{é}partement des arts et lettres - Universit{é} du Qu{é}bec {à}
Chicoutimi}:9‑22. \url{https://doi.org/10.7202/1006723ar}.

\leavevmode\vadjust pre{\hypertarget{ref-averous_verclytte_clarisse_2008}{}}%
Avérous Verclytte, Valérie. 2008. {«~{Clarisse Herrenschmidt, Les trois
{é}critures. Langue, nombre, code}~»}. \emph{Mots. Les langages du
politique}, nᵒ 86 (mars). {ENS {É}ditions}:122‑26.
\url{https://doi.org/10.4000/mots.13872}.

\leavevmode\vadjust pre{\hypertarget{ref-bachimont_du_1999}{}}%
Bachimont, Bruno. 1999. {«~{Du texte {à} l'hypotexte~: les parcours de
la m{é}moire documentaire}~»}. In \emph{{M{é}moire de la technique et
techniques de la m{é}moire}}, 195‑225. {Technologies, id{é}ologies,
pratiques}. {Toulouse}: {{É}r{è}s}.
\url{https://doi.org/10.3917/eres.lenay.1999.01.0195}.

\leavevmode\vadjust pre{\hypertarget{ref-baillehache_digital_2021}{}}%
Baillehache, Jonathan. 2021. {«~The {Digital Reception} of {A Hundred
Thousand Billion Poems}~»}. \emph{Sens Public}, 1.
\url{https://doi.org/10.7202/1089666ar}.

\leavevmode\vadjust pre{\hypertarget{ref-baillet_jugemens_1722}{}}%
Baillet, Adrien. 1722. \emph{Jugemens Des Savans Sur Les Principaux
Ouvrages Des Auteurs}. {Paris}: {C. Moette}.

\leavevmode\vadjust pre{\hypertarget{ref-balsamo_technologies_1996}{}}%
Balsamo, Anne Marie. 1996. \emph{Technologies of the {Gendered Body}:
{Reading Cyborg Women}}. {Durham}: {Duke University Press}.

\leavevmode\vadjust pre{\hypertarget{ref-balzac_comedie_1950}{}}%
Balzac, Honore de. 1979. \emph{{La Comedie humaine}}. Vol. 10.
{Biblioth{è}que de la Pl{é}iade}. {Paris}: {Gallimard}.

\leavevmode\vadjust pre{\hypertarget{ref-balzac_correspondance_2006}{}}%
Balzac, Honoré de. 2006. \emph{Correspondance}. Édité par Roger Pierrot
et Hervé Yon. Vol. 1. Biblioth{è}que de La {Pl{é}iade}. {Paris}:
{Gallimard}.

\leavevmode\vadjust pre{\hypertarget{ref-barad_meeting_2007}{}}%
Barad, Karen Michelle. 2007. \emph{Meeting the {Universe Halfway}:
{Quantum Physics} and the {Entanglement} of {Matter} and {Meaning}}.
{Durham}: {Duke University Press}.

\leavevmode\vadjust pre{\hypertarget{ref-barbier_gutenberg_2012}{}}%
Barbier, Frédéric. 2012. {«~{Gutenberg et l'invention de
l'imprimerie}~»}. In \emph{{Histoire du livre en Occident}}, 83‑102.
{Collection U}. {Paris}: {Armand Colin}.
\url{https://doi.org/10.3917/arco.barbi.2012.01.0083}.

\leavevmode\vadjust pre{\hypertarget{ref-barthes_mort_1968}{}}%
Barthes, Roland. 1968. {«~La Mort de l'auteur~»}. \emph{Manteia}, nᵒ 5.

\leavevmode\vadjust pre{\hypertarget{ref-battles_palimpsest:_2016}{}}%
Battles, Matthew. 2016. \emph{Palimpsest: {A History} of the {Written
Word}}. {W. W. Norton \& Company}.

\leavevmode\vadjust pre{\hypertarget{ref-baudelaire_fusees_2016}{}}%
Baudelaire, Charles. 2016. \emph{{Fus{é}es}}. Édité par André Guyaux.
{Folio}. {Paris}: {Gallimard}.

\leavevmode\vadjust pre{\hypertarget{ref-baum_patchwork_1913}{}}%
Baum, L. Frank. 1913. \emph{The {Patchwork Girl} of {Oz}}. {Chicago}:
{The Reilly \& Birtto Co.}

\leavevmode\vadjust pre{\hypertarget{ref-beaulieu_differenciations_2005}{}}%
Beaulieu, Étienne. 2005. {«~{Diff{é}renciations romanesques :
l'imaginaire technique chez Balzac, Villiers de l'Isle-Adam et Jules
Verne}~»}. \emph{Interm{é}dialit{é}s}, nᵒ 6. {Centre de recherche sur
l'interm{é}dialit{é}}:121‑39. \url{https://doi.org/10.7202/1005510ar}.

\leavevmode\vadjust pre{\hypertarget{ref-benabou_quarante_2000}{}}%
Benabou, Marcel. 2000. {«~{Quarante si{è}cles d'Oulipo}~»}. \emph{Raison
pr{é}sente} 134 (1). {Pers{é}e - Portail des revues scientifiques en
SHS}:71‑90. \url{https://doi.org/10.3406/raipr.2000.3611}.

\leavevmode\vadjust pre{\hypertarget{ref-benjamin_illuminations_1968}{}}%
Benjamin, Walter. 1968. \emph{Illuminations: {Essays} and
{Reflections}}. Édité par Hannah Arendt. Traduit par Harry Zohn. {Boston
; New York}: {Mariner Books, Houghton Mifflin Harcourt}.

\leavevmode\vadjust pre{\hypertarget{ref-benjamin_oeuvre_2007}{}}%
Benjamin, Walter. 2007. \emph{{L'{œ}uvre d'art {à} l'{é}poque de sa
reproductibilit{é} technique: version de 1939}}. Édité par Maurice de
Gandillac et Seloua Luste Boulbina. Traduit par Lambert Dousson.
{Folioplus}. {Paris}: {Gallimard}.

\leavevmode\vadjust pre{\hypertarget{ref-benn_den_1966}{}}%
Benn, Gottfried. 1966. \emph{{Den Traum alleine tragen}}. Erstausgabe.
edition. {Limes}.

\leavevmode\vadjust pre{\hypertarget{ref-bennes_klara_2023}{}}%
Bennes, Crystal. 2023. {«~{Klara et la bombe}~»}. Traduit par Guillaume
Heuguet. \emph{T{è}que} 3 (1). {Audimat {É}ditions}:10‑38.

\leavevmode\vadjust pre{\hypertarget{ref-bennett_vibrant_2010}{}}%
Bennett, Jane. 2010. \emph{Vibrant {Matter}: A {Political Ecology} of
{Things}}. {Durham}: {Duke University Press}.

\leavevmode\vadjust pre{\hypertarget{ref-bens_genese_2005}{}}%
Bens, Jacques, et Jacques Duchateau. 2005. \emph{Gen{è}se de l'{Oulipo}
: 1960-1963}. {Bordeaux}: {Le Castor Astral}.

\leavevmode\vadjust pre{\hypertarget{ref-bernanos_france_2015}{}}%
Bernanos, Georges. 2015. \emph{{La France contre les robots}}. {Les
Inattendus}. {B{è}gles}: {Le Castor Astral}.

\leavevmode\vadjust pre{\hypertarget{ref-bernardot_plongee_2018}{}}%
Bernardot, Marc. 2018. {«~{Plong{é}e dans les m{é}taphores et
repr{é}sentations liquides de la soci{é}t{é} num{é}rique}~»}.
\emph{Netcom. R{é}seaux, communication et territoires}, nᵒ 32-1/2.
{Netcom Association}:29‑60. \url{https://doi.org/10.4000/netcom.2886}.

\leavevmode\vadjust pre{\hypertarget{ref-beyerlen_lustige_1909}{}}%
Beyerlen, Angelo. 1909. {«~Eine Lustige {Geschichte} von {Blinden}
Usw.~»} \emph{Schreibmaschinen-Zeitung Hamburg} 138:362‑63.

\leavevmode\vadjust pre{\hypertarget{ref-birkets_gutenberg_2006}{}}%
Birkets, Sven. 2006. \emph{{The Gutenberg Elegies: The Fate of Reading
in an Electronic Age}}. {New York}: {North Point Press}.

\leavevmode\vadjust pre{\hypertarget{ref-bisenius-penin_50_2012}{}}%
Bisenius-Penin, Carole, et André Petitjean. 2012. \emph{{50 ans
d'OULIPO~: de la contrainte {à} l'{œ}uvre}}. {La Licorne} 100. {Rennes}:
{Presses universitaires de Rennes}.

\leavevmode\vadjust pre{\hypertarget{ref-blanchard_les_1993}{}}%
Blanchard, Alain. 1993. {«~{Les papyrus litt{é}raires grecs extraits de
cartonnages : {é}tudes de bibliologie}~»}. In \emph{{Ancient and
Medieval book materials and techniques : Erice, 18-25 september 1992 /
edited by Marilena Maniaci, Paola F. Munaf{Ã}{\(^{2}\)}}}. {Biblioteca
Apostolica Vaticana}.

\leavevmode\vadjust pre{\hypertarget{ref-bliven_wonderful_1954}{}}%
Bliven, Bruce. 1954. \emph{{Wonderful Writing Machine}}. First Edition.
{Random House{\textasciitilde{}}childrens}.

\leavevmode\vadjust pre{\hypertarget{ref-bloomfield_machines_2016}{}}%
Bloomfield, Camille, et Hélène Campaignolle-Catel. 2016. {«~{Machines
litt{é}raires, machines num{é}riques : l'Oulipo et l'informatique}~»}.
In \emph{{Oulipo, mode d'emploi}}, 15. {Honor{é} Champion {é}diteur}.

\leavevmode\vadjust pre{\hypertarget{ref-bohac_mallarme_2021}{}}%
Bohac, Barbara. 2021. {«~{Mallarm{é} et l'esth{é}tique du livre}~»}. In
\emph{{L'Esth{é}tique du livre}}, édité par Alain Milon et Marc
Perelman, 149‑64. {Livre et soci{é}t{é}}. {Nanterre}: {Presses
universitaires de Paris Nanterre}.
\url{https://doi.org/10.4000/books.pupo.1883}.

\leavevmode\vadjust pre{\hypertarget{ref-bok_xenotext._2015}{}}%
Bök, Christian. 2015. \emph{The Xenotext. {Book} 1}. {Toronto}: {Coach
House Books}.

\leavevmode\vadjust pre{\hypertarget{ref-bolter_remediation:_2003}{}}%
Bolter, Jay David, et Richard Grusin. 2003. \emph{Remediation:
{Understanding New Media}}. 6. Nachdr. {Cambridge, Mass.}: {MIT Press}.

\leavevmode\vadjust pre{\hypertarget{ref-bon_pas_2003}{}}%
Bon, François. 2003. {«~Pas Besoin de La Notion d'oeuvre~»}.

\leavevmode\vadjust pre{\hypertarget{ref-bon_flaubert_2021}{}}%
Bon, François. 2021. {«~Flaubert {É}crivain Technologique \textbar{} Les
Num{é}riques \#11~»}.

\leavevmode\vadjust pre{\hypertarget{ref-bon_numeriques_2022}{}}%
Bon, François. 2022. {«~\#num{é}riques \textbar~Litt{é}rature \&
Intelligence Artificielle, La Bagarre a Commenc{é} !~»}

\leavevmode\vadjust pre{\hypertarget{ref-bonaccorsi_fantasmagories_2012}{}}%
Bonaccorsi, Julia. 2012. {«~{Fantasmagories de l'{é}cran}~»}. Thèse de
doctorat, Celsa - Universit{é} Paris Sorbonne.

\leavevmode\vadjust pre{\hypertarget{ref-bonnet_pour_2017}{}}%
Bonnet, Gilles. 2017. \emph{{Pour une po{é}tique num{é}rique :
litt{é}rature et Internet}}. {Collection Savoir lettres}. {Paris}:
{Hermann}.

\leavevmode\vadjust pre{\hypertarget{ref-borges_libro_2008}{}}%
Borges, Jorge Luis. 2008. \emph{{El libro de arena}}. 12. reimpr.
{Biblioteca Borges} 3. {Madrid}: {Alianza Ed}.

\leavevmode\vadjust pre{\hypertarget{ref-borges_livre_2018}{}}%
Borges, Jorge Luis. 2018. \emph{{Le livre de sable}}. Édité par
Françoise Rosset. Nouvelle {é}d. {Collection Folio} 1461. {Paris}:
{Gallimard}.

\leavevmode\vadjust pre{\hypertarget{ref-bosanquet_henry_2006}{}}%
Bosanquet, Theodora. 2006. \emph{Henry {James} at {Work}}. Édité par
Lyall Powers. {Ann Arbor, MI}: {University of Michigan Press}.
\url{https://doi.org/10.3998/mpub.98996}.

\leavevmode\vadjust pre{\hypertarget{ref-bouchardon_valeur_2014}{}}%
Bouchardon, Serge. 2014. \emph{La Valeur Heuristique de La Litt{é}rature
Num{é}rique}. Collection {Cultures} Num{é}riques. {Paris}: {Hermann}.

\leavevmode\vadjust pre{\hypertarget{ref-bouchardon_formules_2006}{}}%
Bouchardon, Serge, Eduardo Kac, et Jean-Pierre Balpe. 2006.
\emph{{Formules \# 10 : La Litt{é}rature num{é}rique et caetera}}.
{Paris}: {Agn{è}s Vienot {é}diteur}.

\leavevmode\vadjust pre{\hypertarget{ref-bouchet_skin_2020}{}}%
Bouchet, Marie. 2020. {«~The {SKIN Project} by {Shelley Jackson}. {The
Tattooed Text} as a {Mortal Work} of {Art}.~»} \emph{La Peaulogie -
Revue de Sciences Sociales Et Humaines Sur Les Peaux}, nᵒ 4:145.

\leavevmode\vadjust pre{\hypertarget{ref-bousseyroux_lacan_2015}{}}%
Bousseyroux, Michel. 2015. {«~{Lacan avec Mallarm{é}~: l'Action
restreinte de l'analyste}~»}. \emph{L'en-je lacanien} 24 (1).
{Toulouse}: {{É}r{è}s}:57‑71.
\url{https://doi.org/10.3917/enje.024.0057}.

\leavevmode\vadjust pre{\hypertarget{ref-bradbury_fahrenheit_1953}{}}%
Bradbury, Ray. 1953. \emph{Fahrenheit 451}. {Ballantine Books}.

\leavevmode\vadjust pre{\hypertarget{ref-bradshaw__work_2013}{}}%
Bradshaw, Bethany. 2013. {«~The {Work} of {Art} in the {Age} of
{Technological Translation}: {The Materiality} of {Anne Carson}'s
{Nox}~»}. \emph{Free Verse}, nᵒ 23.

\leavevmode\vadjust pre{\hypertarget{ref-braidotti_posthuman_2019}{}}%
Braidotti, Rosi. 2019. \emph{Posthuman {Knowledge}}. {Medford, MA}:
{Polity}.

\leavevmode\vadjust pre{\hypertarget{ref-brubaker_palimpsest_1987}{}}%
Brubaker, Leslie. 1987. {«~Palimpsest~»}. \emph{Dictionary of the Middle
Ages} 9. {New York}: {Charles Scribner's Sons}:355.

\leavevmode\vadjust pre{\hypertarget{ref-bruck_schicksale_1930}{}}%
Brück, Anita. 1930. \emph{{Schicksale hinter Schreibmaschinen}}.
{Sieben-St{ä}be-Verlag}.

\leavevmode\vadjust pre{\hypertarget{ref-bruck_mademoiselle_1950}{}}%
Brück, Anita. 1950. \emph{{Mademoiselle Br{ü}ckner, dactylo}}. Traduit
par Raymond Henry. {Busson}.

\leavevmode\vadjust pre{\hypertarget{ref-bruzina_art_1982}{}}%
Bruzina, Ronald. 1982. {«~Art and {Architecture}, {Ancient} and
{Modern}~»}. \emph{Research in Philosophy \& Technology} 5:163‑87.

\leavevmode\vadjust pre{\hypertarget{ref-burghagen_schreibmaschine_2003}{}}%
Burghagen, Otto. 2003. \emph{{Die Schreibmaschine.: Ein praktisches
Handbuch enthaltend Illustrierte Beschreibung aller gangbaren
Schreibmaschinen.}} {Kunstgrafik Dingwerth}.

\leavevmode\vadjust pre{\hypertarget{ref-bustarret_couper_2010}{}}%
Bustarret, Claire. 2010. {«~Couper, Coller Dans Les Manuscrits de
Travail Du {XVIIIe} Au {XXe} Si{è}cle~»}. In \emph{Lieux de Savoir. 2:
Les Mains de l'intellect}, 353‑75. {Paris}: {Albin Michel}.

\leavevmode\vadjust pre{\hypertarget{ref-butler_humain_2005}{}}%
Butler, Judith. 2005. \emph{{Humain, inhumain. Le travail critique des
normes : entretiens}}. Traduit par Christine VIvier et Jérôme Vidal.
{Paris}: {{É}ditions Amsterdam}.

\leavevmode\vadjust pre{\hypertarget{ref-butler_defaire_2006}{}}%
Butler, Judith. 2006. \emph{{D{é}faire le genre}}. {Paris}: {{É}ditions
Amsterdam}.

\leavevmode\vadjust pre{\hypertarget{ref-butler_matter_2011}{}}%
Butler, Shane. 2011. \emph{The {Matter} of the {Page}: {Essays} in
{Search} of {Ancient} and {Medieval Authors}}. Wisconsin Studies in
Classics. {Madison}: {University of Wisconsin Press}.

\leavevmode\vadjust pre{\hypertarget{ref-calle-gruber_butor_1998}{}}%
Calle-Gruber, Mireille. 1998. \emph{Butor et l'{Am{é}rique} : Colloque
de {Queen}'s {University}}. Édité par Queen's University (Kingston,
Ont.). Collection {Trait} d'union. {Paris, France}: {L'Harmattan}.

\leavevmode\vadjust pre{\hypertarget{ref-carrier-lafleur_invention_2018}{}}%
Carrier-Lafleur, Thomas, André Gaudreault, Servanne Monjour, et Marcello
Vitali-Rosati. 2018. {«~{L'Invention litt{é}raire des m{é}dias}~»}.
\emph{Sens public}, nᵒ 1305 (mars).

\leavevmode\vadjust pre{\hypertarget{ref-carson_nox_2010}{}}%
Carson, Anne. 2010. \emph{Nox}. {New York}: {New Directions}.

\leavevmode\vadjust pre{\hypertarget{ref-casilli_en_2019}{}}%
Casilli, Antonio A. 2019. \emph{En Attendant Les Robots: Enqu{ê}te Sur
Le Travail Du Clic}. La {Couleur} Des Id{é}es. {Paris XIXe}: {{É}ditions
du Seuil}.

\leavevmode\vadjust pre{\hypertarget{ref-casilli_il_2021}{}}%
Casilli, Antonio A. 2021. {«~{Il n'y a pas d'intelligence artificielle,
il n'y a que le travail du clic de quelqu'un d'autre}~»}. In, 33.
{{É}ditions du D{é}tours}.

\leavevmode\vadjust pre{\hypertarget{ref-cendrars_prose_1913}{}}%
Cendrars, Blaise, et Sonia Delauney. 1913. \emph{La Prose Du
{Transsib{é}rien} et de La {Petite Jehanne} de {France}}. {{É}dition des
Hommes Nouveaux}.

\leavevmode\vadjust pre{\hypertarget{ref-cendrars_blaise_2006}{}}%
Cendrars, Miriam. 2006. \emph{{Blaise Cendrars : la vie, le verbe,
l'{é}criture}}. {É}d. revue, corrig{é}e, augment{é}e. {Paris}:
{Deno{ë}l}.

\leavevmode\vadjust pre{\hypertarget{ref-certeau_arts_2010}{}}%
Certeau, Michel de. 2010. \emph{{Arts de faire}}. Édité par Luce Giard.
Nouvelle {é}d. {L' invention du quotidien / Michel de Certeau} 1.
{Paris}: {Gallimard}.

\leavevmode\vadjust pre{\hypertarget{ref-chartier_culture_1997}{}}%
Chartier, Roger. 1997. \emph{{Culture {é}crite et soci{é}t{é} : l'ordre
des livres (XIVe - XVIIIe si{è}cle)}}. {Biblioth{è}que Albin Michel
Histoire}. {Paris}: {Albin Michel}.

\leavevmode\vadjust pre{\hypertarget{ref-chartier_lecrit_2005}{}}%
Chartier, Roger. 2005a. {«~{De l'{é}crit sur l'{é}cran. {É}criture
{é}lectronique et ordre du discours}~»}. In \emph{{Les {é}critures
d'{é}cran : histoire, pratiques et espace sur le Web}}.
{Aix-en-Provence}: {Hypotheses}.

\leavevmode\vadjust pre{\hypertarget{ref-chartier_inscrire_2005}{}}%
Chartier, Roger. 2005b. \emph{Inscrire et Effacer : Culture {É}crite et
Litt{é}rature ({XIe-XVIIIe} Si{è}cle)}. Hautes {É}tudes. {Paris}:
{Gallimard : Seuil}.

\leavevmode\vadjust pre{\hypertarget{ref-chartier_lecrit_2006}{}}%
Chartier, Roger. 2006. {«~{L'{é}crit sur l'{é}cran. Ordre du discours,
ordre des livres et mani{è}res de lire}~»}. \emph{Entreprises et
histoire} 43 (2):15. \url{https://doi.org/10.3917/eh.043.0015}.

\leavevmode\vadjust pre{\hypertarget{ref-chenoweth_faute_2020}{}}%
Chenoweth, Katie. 2020. {«~{Faute de frappe : Derrida dactylo}~»}.
\emph{Philosophiques} 47 (2). {Soci{é}t{é} de philosophie du
Qu{é}bec}:333‑49. \url{https://doi.org/10.7202/1075127ar}.

\leavevmode\vadjust pre{\hypertarget{ref-choderlos_de_laclos_les_1782}{}}%
Choderlos de Laclos, Pierre Ambroise François. 1782. \emph{{Les liaisons
dangereuses}}. Édité par Michel Delon. {Le livre de poche} 354. {Paris}:
{Librairie G{é}n{é}rale Fran{ç}aise}.

\leavevmode\vadjust pre{\hypertarget{ref-christin_image_1995}{}}%
Christin, Anne-Marie. 1995. \emph{L'image {É}crite, Ou, {La} D{é}raison
Graphique}. Id{é}es et Recherches. {Paris}: {Flammarion}.

\leavevmode\vadjust pre{\hypertarget{ref-christin_les_1999}{}}%
Christin, Anne-Marie. 1999. {«~{Les origines de l'{é}criture : image,
signe, trace}~»}. \emph{Le D{é}bat} 106 (4):28.
\url{https://doi.org/10.3917/deba.106.0028}.

\leavevmode\vadjust pre{\hypertarget{ref-christin_poetique_2009}{}}%
Christin, Anne-Marie. 2009. \emph{{Po{é}tique du blanc : vide et
intervalle dans la civilisation de l'alphabet}}. Nouvelle {é}d. revue et
augment{é}e. {Essais d'art et de philosophie}. {Paris}: {Vrin}.

\leavevmode\vadjust pre{\hypertarget{ref-citton_hermeneutique_2015}{}}%
Citton, Yves. 2015. {«~{Herm{é}neutique et (re)m{é}diation~: vers des
{é}tudes de media compar{é}s~?}~»} \emph{Critique} 817--818 (6-7).
{Paris}: {{É}ditions de Minuit}:569‑81.
\url{https://doi.org/10.3917/criti.817.0569}.

\leavevmode\vadjust pre{\hypertarget{ref-cocteau_du_2003}{}}%
Cocteau, Jean. 2003. \emph{{Du cin{é}matographe}}. Édité par André
Bernard. Nouvelle {[}3.{]} {é}d. revue et augment{é}e. {Monaco}: {Ed. du
Rocher}.

\leavevmode\vadjust pre{\hypertarget{ref-coffee_intertextuality_2012}{}}%
Coffee, Neil, Jean-Pierre Koenig, Shakthi Poornima, Roelant Ossewaarde,
Christopher Forstall, et Sarah Jacobson. 2012. {«~Intertextuality in the
{Digital Age}~»}. \emph{Transactions of the American Philological
Association} 142 (2):383‑422.
\url{https://doi.org/10.1353/apa.2012.0010}.

\leavevmode\vadjust pre{\hypertarget{ref-collet_disparition_2004}{}}%
Collet, Isabelle. 2004. {«~{La disparition des filles dans les {é}tudes
d'informatique : les cons{é}quences d'un changement de
repr{é}sentation}~»}. \emph{Carrefours de l'education} n{\textdegree} 17
(1):42‑56.

\leavevmode\vadjust pre{\hypertarget{ref-collet_auto-engendrement_2009}{}}%
Collet, Isabelle. 2009. {«~{L'auto-engendrement des informaticiens :
comment supprimer la diff{é}rence des sexes gr{â}ce {à} un mode de
reproduction fantasm{é}e}~»}. \emph{Sextant}, nᵒ 27:207‑20.

\leavevmode\vadjust pre{\hypertarget{ref-collet_les_2017}{}}%
Collet, Isabelle. 2017. {«~{Les informaticiennes : de la dominance de
classe aux discriminations de sexe}~»}. \emph{1024 {\textendash}
Bulletin de la soci{é}t{é} informatique de France}, nᵒ 2:25.

\leavevmode\vadjust pre{\hypertarget{ref-collet_les_2019}{}}%
Collet, Isabelle. 2019. \emph{{Les oubli{é}es du num{é}rique}}. {Paris}:
{le Passeur {é}diteur}.

\leavevmode\vadjust pre{\hypertarget{ref-courant_ecrire_2016}{}}%
Courant, Elsa. 2016. {«~{É}crire {Au} {\guillemotleft} {Folio Du Ciel}
{\guillemotright} : {Le Mod{è}le De La Constellation Dans} "{Un Coup De
D{é}s}" {De Mallarm{é}}~»}. \emph{Revue d'Histoire litt{é}raire de la
France} 116 (4). {Presses Universitaires de France}:869‑91.
\url{https://www.jstor.org/stable/44516157}.

\leavevmode\vadjust pre{\hypertarget{ref-crozat_ements_2011}{}}%
Crozat, Stéphane, Bruno Bachimont, Isabelle Cailleau, Serge Bouchardon,
et Ludovic Gaillard. 2011. {«~{É}l{é}ments Pour Une Th{é}orie
Op{é}rationnelle de l'{é}criture Num{é}rique~»}. \emph{Document
num{é}rique} 14 (3):9‑33. \url{https://doi.org/10.3166/dn.14.3.9-33}.

\leavevmode\vadjust pre{\hypertarget{ref-current_typewriter_1988}{}}%
Current, Richard Nelson. 1988. \emph{The {Typewriter} and the {Men Who
Made} It}. 2nd ed. {Arcadia, CA}: {Post-Era Books}.

\leavevmode\vadjust pre{\hypertarget{ref-cyrulnik_sous_2010}{}}%
Cyrulnik, Boris. 2010. \emph{{Sous le signe du lien : une histoire
naturelle de l'attachement}}. Nouvelle {é}d. {Paris}: {A.
Fayard-Pluriel}.

\leavevmode\vadjust pre{\hypertarget{ref-de_iulio_lev_2003}{}}%
De Iulio, Simona. 2003. {«~{Lev Manovich, The Language of New Media.
Cambridge, Massachusetts, London, England, The MIT Press, 2001, 354
p.}~»} \emph{Questions de communication}, nᵒ 4 (décembre). {Presses
universitaires de Nancy}:473‑75.

\leavevmode\vadjust pre{\hypertarget{ref-debbaut_lissitzky_1991}{}}%
Debbaut, Jan. 1991. \emph{El {Lissitzky}: 1890-1941: {Architect},
{Painter}, {Photographer}, {Typographer}}. {Paris}: {Thames \& Hudson}.

\leavevmode\vadjust pre{\hypertarget{ref-deleuze_francis_2002}{}}%
Deleuze, Gilles. 2002. \emph{{Francis Bacon, logique de la sensation}}.
{L'ordre philosophique}. {Paris}: {{É}ditions du Seuil}.

\leavevmode\vadjust pre{\hypertarget{ref-deleuze_pourparlers_2003}{}}%
Deleuze, Gilles. 2003. \emph{{Pourparlers 1972-1990}}. {Reprise}.
{Paris}: {{É}ditions de Minuit}.

\leavevmode\vadjust pre{\hypertarget{ref-depaule_fable_2014}{}}%
Depaule, Jean-Charles. 2014. {«~{La fable d'une fabrique. Ponge et son
pr{é}}~»}. \emph{Gradhiva. Revue d'anthropologie et d'histoire des
arts}, nᵒ 20 (octobre). {Mus{é}e du quai Branly}:22‑47.
\url{https://doi.org/10.4000/gradhiva.2821}.

\leavevmode\vadjust pre{\hypertarget{ref-derose_what_1997}{}}%
DeRose, Steven J., David G. Durand, Elli Mylonas, et Allen H. Renear.
1997. {«~What {Is Text}, {Really}?~»} \emph{SIGDOC Asterisk J. Comput.
Doc.} 21 (3):1‑24. \url{https://doi.org/10.1145/264842.264843}.

\leavevmode\vadjust pre{\hypertarget{ref-derrida_grammatologie_1967}{}}%
Derrida, Jacques. 1967. \emph{De La Grammatologie}. Collection
"{Critique}". {Paris}: {{É}dtions de Minuit}.

\leavevmode\vadjust pre{\hypertarget{ref-derrida_paper_2005}{}}%
Derrida, Jacques. 2005. \emph{Paper {Machine}}. Traduit par Rachel
Bowlby. 1st edition. {Stanford, Calif}: {Stanford University Press}.

\leavevmode\vadjust pre{\hypertarget{ref-derrida_animal_2006}{}}%
Derrida, Jacques. 2006. \emph{L'animal Que Donc Je Suis}. Édité par
Marie-Louise Mallet. Collection {La} Philosophie En Effet. {Paris}:
{Galil{é}e}.

\leavevmode\vadjust pre{\hypertarget{ref-descartes_oeuvres_2010}{}}%
Descartes, René. 2010. \emph{{{Œ}uvres philosophiques}}. Édité par
Ferdinand Alquié et Denis Moreau. {É}d. corrig{é}e. {Textes de
philosophie} 4. {Paris}: {{É}ditions Classiques Garnier}.

\leavevmode\vadjust pre{\hypertarget{ref-despoix_presentation_2005}{}}%
Despoix, Philippe, et Yvonne Spielmann. 2005. {«~{Pr{é}sentation}~»}.
\emph{Interm{é}dialit{é}s : histoire et th{é}orie des arts, des lettres
et des techniques / Intermediality: History and Theory of the Arts,
Literature and Technologies}, nᵒ 6. {Centre de recherche sur
l'interm{é}dialit{é}}:9‑11. \url{https://doi.org/10.7202/1005502ar}.

\leavevmode\vadjust pre{\hypertarget{ref-diestchy__2020}{}}%
Diestchy, Mireille. 2020. {«~{Le {\guillemotleft} faire
{\guillemotright} au c{œ}ur de l'initiation {à} l'{\guillemotleft} art
de l'enqu{ê}te {\guillemotright}}~»}. \emph{Terrains/Th{é}ories}, nᵒ 12
(décembre). {Presses universitaires de Paris Nanterre}.

\leavevmode\vadjust pre{\hypertarget{ref-dillon_palimpsest:_2007}{}}%
Dillon, Sarah. 2007. \emph{The {Palimpsest}: {Literature}, {Criticism},
{Theory}}. Continuum Literary Studies Series. {London ; New York}:
{Continuum}.

\leavevmode\vadjust pre{\hypertarget{ref-dobretsov_relation_1975}{}}%
Dobretsov, G. E., I. G. Kharitonenkov, V. E. Mishiev, et Iu A.
Vladimirov. 1975.
{«~\href{https://www.ncbi.nlm.nih.gov/pubmed/94}{{Relation between
fluorescence and circular dichroism of the complex of the fluorescence
probe 4-dimethylaminochalcone with serum albumin}}~»}. \emph{Biofizika}
20 (4):581‑85.

\leavevmode\vadjust pre{\hypertarget{ref-donovan_line_2003}{}}%
Donovan, Matt. 2003. {«~Line~»}.

\leavevmode\vadjust pre{\hypertarget{ref-doueihi_livre_2010}{}}%
Doueihi, Milad. 2010. {«~{Le livre {à} l'heure du num{é}rique~: objet
f{é}tiche, objet de r{é}sistance}~»}. In \emph{{Read/Write Book : Le
livre inscriptible}}, édité par Marin Dacos, 95‑103. {Read/Write Book}.
{Marseille}: {OpenEdition Press}.

\leavevmode\vadjust pre{\hypertarget{ref-doueihi_pour_2011}{}}%
Doueihi, Milad. 2011. \emph{{Pour un humanisme num{é}rique}}. {Paris}:
{{É}ditions du Seuil}.

\leavevmode\vadjust pre{\hypertarget{ref-drucker_century_2004}{}}%
Drucker, Johanna. 2004. \emph{The {Century Of Artists}' {Books}}. 2nd
Revised ed. edition. {New York City}: {Granary Books}.

\leavevmode\vadjust pre{\hypertarget{ref-during_modern_2002}{}}%
During, Simon. 2002. \emph{Modern {Enchantments}: The {Cultural Power}
of {Secular Magic}}. {Cambridge (Mass.) London}: {Harvard University
Press}.

\leavevmode\vadjust pre{\hypertarget{ref-dworkin_parse_2008}{}}%
Dworkin, Craig. 2008. \emph{{Parse}}. {Atelos}.

\leavevmode\vadjust pre{\hypertarget{ref-dyens_enfanter_2012}{}}%
Dyens, Ollivier. 2012. \emph{{Enfanter l'inhumain. Le refus du vivant}}.
{{É}ditions Triptyque}.

\leavevmode\vadjust pre{\hypertarget{ref-eberle-sinatra_pratiques_2014}{}}%
Eberle-Sinatra, Michael, et Marcello Vitali Rosati. 2014.
\emph{{Pratiques de l'{é}dition num{é}rique}}. {Parcours num{é}riques}.
{Montr{é}al}: {Les Presses de l'Universit{é} de Montr{é}al}.

\leavevmode\vadjust pre{\hypertarget{ref-eco_guerre_1985}{}}%
Eco, Umberto. 1985. \emph{{La guerre du faux}}. Traduit par Myriam
Tanant. {Paris}: {Grasset}.

\leavevmode\vadjust pre{\hypertarget{ref-eco_travels_1990}{}}%
Eco, Umberto. 1990. \emph{Travels in {Hyperreality}: {Essays}}. Traduit
par William Weaver. A {Harvest} Book. {San Diego, Calif.}: {Harcourt
Brace}.

\leavevmode\vadjust pre{\hypertarget{ref-eco_how_2015}{}}%
Eco, Umberto. 2015a. \emph{How to {Write} a {Thesis}}. Traduit par
Caterina Mongiat Farina. {Cambridge (Mass.) London}: {MIT Press}.

\leavevmode\vadjust pre{\hypertarget{ref-eco_loeuvre_2015}{}}%
Eco, Umberto. 2015b. \emph{{L'{œ}uvre ouverte}}. Édité par Chantal Roux
de Bézieux. {Points} 107. {Paris}: {{É}ditions Points}.

\leavevmode\vadjust pre{\hypertarget{ref-eco_nom_2022}{}}%
Eco, Umberto. 2022. \emph{{Le nom de la rose}}. Édité par Jean-Noël
Schifano. Traduit par Mario Andreose. Nouvelle {é}d. augment{é}e des
dessins et notes pr{é}paratoires de l'auteur. {Paris}: {Bernard
Grasset}.

\leavevmode\vadjust pre{\hypertarget{ref-ermolieff_pillow_2012}{}}%
Ermolieff, Anne. 2012. {«~{Pillow Book ou l'expression de deux
fascinations sans limites : la chair et la calligraphie}~»}.
\emph{Savoirs et clinique} n{\textdegree} 15 (1):151‑56.

\leavevmode\vadjust pre{\hypertarget{ref-fan_value_2018}{}}%
Fan, Lai-Tze. 2018. {«~On the {Value} of {Narratives} in a {Reflexive
Digital Humanities}~»}. \emph{Digital Studies / Le Champ Num{é}rique} 8
(1). {Open Library of Humanities}.
\url{https://doi.org/10.16995/dscn.285}.

\leavevmode\vadjust pre{\hypertarget{ref-fan_reverse_2023}{}}%
Fan, Lai-Tze. 2023. {«~Reverse {Engineering} the {Gendered Design} of
{Amazon}'s {Alexa}: {Methods} in {Testing Closed-Source Code} in {Grey}
and {Black Box Systems}~»}. \emph{Digital Humanities Quarterly} 017 (2).

\leavevmode\vadjust pre{\hypertarget{ref-10.2307ux2f43801959}{}}%
Fassié, Pierre. 1984. {«~Interpr{é}tations Du Cryptogramme de La
"{Physiologie} Du Mariage" ({M{é}ditation XXV}) de {Balzac}~»}.
\emph{Romance Notes} 24 (3):249‑53.
\url{https://www.jstor.org/stable/43801959}.

\leavevmode\vadjust pre{\hypertarget{ref-faure_histoire_1991}{}}%
Faure, Elie, et Martine Courtois. 1991. \emph{Histoire de l'art.
{L}'esprit Des Formes}. Collection {Folio}/Essais 176-177. {Paris}:
{Gallimard}.

\leavevmode\vadjust pre{\hypertarget{ref-felici_manuel_2003}{}}%
Felici, James. 2003. \emph{{Le manuel complet de typographie}}.
{Peachpitt Press}.

\leavevmode\vadjust pre{\hypertarget{ref-ferrara_grande_2019}{}}%
Ferrara, Silvia. 2019. \emph{La Grande Invenzione: Storia Del Mondo in
Nove Scritture Misteriose}. Prima edizione in "Varia.".
Varia/{Feltrinelli}. {Milano}: {Feltrinelli}.

\leavevmode\vadjust pre{\hypertarget{ref-ferrara_fabuleuse_2021}{}}%
Ferrara, Silvia. 2021. \emph{{La fabuleuse histoire de l'invention de
l'{é}criture}}. Édité par Jacques Dalarun. {Paris}: {{É}ditions du
Seuil}.

\leavevmode\vadjust pre{\hypertarget{ref-ferrara_invention_2021}{}}%
Ferrara, Silvia, et La manufacture des idées. 2021. {«~{L'invention de
l'{é}criture}~»}.

\leavevmode\vadjust pre{\hypertarget{ref-ferraris_ame_2014}{}}%
Ferraris, Maurizio. 2014. \emph{{{Â}me et iPad}}. Traduit par Hélène
Beauchet. \emph{{Â}me et iPad}. {Parcours num{é}rique}. {Montr{é}al}:
{Presses de l'Universit{é} de Montr{é}al}.
\url{https://doi.org/10.4000/books.pum.289}.

\leavevmode\vadjust pre{\hypertarget{ref-fiormonte_digital_2015}{}}%
Fiormonte, Domenico, Teresa Numerico, et Francesca Tomasi. 2015.
\emph{The {Digital Humanist}: {A Critical Inquiry}}. Traduit par Desmond
Schmidt. First published. Digital Humanities / {Global} Computing.
{Brooklyn, NY}: {punctum books}.

\leavevmode\vadjust pre{\hypertarget{ref-flusser_vom_1997}{}}%
Flusser, Vilém. 1997. \emph{{Vom Stand der Dinge: eine kleine
Philosophie des Designs}}. Édité par Fabian Wurm. {G{ö}ttingen}: {Steidl
Verl}.

\leavevmode\vadjust pre{\hypertarget{ref-flusser_shape_1999}{}}%
Flusser, Vilém. 1999. \emph{The {Shape} of {Things}: {A Philosophy} of
{Design}}. {London}: {Reaktion}.

\leavevmode\vadjust pre{\hypertarget{ref-flusser_petite_2002}{}}%
Flusser, Vilém. 2002. \emph{{Petite philosophie du design}}. Traduit par
Claude Maillard. {Belval}: {Circ{é}}.

\leavevmode\vadjust pre{\hypertarget{ref-flusser_pour_2004}{}}%
Flusser, Vilém. 2004. \emph{{Pour une philosophie de la photographie}}.
{Belval}: {Circ{é}}.

\leavevmode\vadjust pre{\hypertarget{ref-foucault_quest_1969}{}}%
Foucault, Michel. 1969. {«~Qu'est Ce Qu'un Auteur ?~»} \emph{Bulletin de
la Soci{é}t{é} fran{ç}aise de philosophie} 63 (3):73‑104.

\leavevmode\vadjust pre{\hypertarget{ref-foucault_archeologie_2008}{}}%
Foucault, Michel. 2008. \emph{{L'arch{é}ologie du savoir}}. {Tel} 354.
{Paris}: {Gallimard}.

\leavevmode\vadjust pre{\hypertarget{ref-galison_einsteins_2003}{}}%
Galison, Peter. 2003. \emph{Einstein's {Clocks} and {Poincar{é}}'s
{Maps}: {Empires} of {Time}}. 1st ed. {New York}: {W.W. Norton}.

\leavevmode\vadjust pre{\hypertarget{ref-galloway_interface_2013}{}}%
Galloway, Alexander R. 2013. \emph{The {Interface Effect}}. {New York,
NY}: {John Wiley \& Sons}.

\leavevmode\vadjust pre{\hypertarget{ref-galloway_excommunication:_2014}{}}%
Galloway, Alexander R., Eugene Thacker, et McKenzie Wark. 2014.
\emph{Excommunication: {Three Inquiries} in {Media} and {Mediation}}.
Trios. {Chicago ; London}: {The University of Chicago Press}.

\leavevmode\vadjust pre{\hypertarget{ref-gamble_what_2019}{}}%
Gamble, Christopher N., Joshua S. Hanan, et Thomas Nail. 2019. {«~What
Is {New Materialism}?~»} \emph{Angelaki-Journal of The Theoretical
Humanities} 24 (6):111‑34.
\url{https://doi.org/10.1080/0969725X.2019.1684704}.

\leavevmode\vadjust pre{\hypertarget{ref-gardey_au_2009}{}}%
Gardey, Delphine. 2009. {«~Au C{œ}ur {à} Corps Avec Le "{Manifeste
Cyborg}" de {Donna Haraway}~»}. \emph{Esprit}, nᵒ 353 (3/4). {Editions
Esprit}:208‑17. \url{https://www.jstor.org/stable/24268060}.

\leavevmode\vadjust pre{\hypertarget{ref-genette_palimpseste._1982}{}}%
Genette, Gérard. 1982. \emph{Palimpseste. {La Litt{é}rature} Au Second
Degr{é}}. {Paris}: {{É}ditions du Seuil}.

\leavevmode\vadjust pre{\hypertarget{ref-genette_loeuvre_1994}{}}%
Genette, Gérard. 1994. \emph{L'{œ}uvre de l'art}. Collection
{Po{é}tique}. {Paris}: {{É}ditions du Seuil}.

\leavevmode\vadjust pre{\hypertarget{ref-gervais_recits_1990}{}}%
Gervais, Bertrand. 1990. \emph{{R{é}cits et actions : pour une th{é}orie
de la lecture}}. {Collection l'univers des discours}. {Longueuil}: {Le
Pr{é}ambule}.

\leavevmode\vadjust pre{\hypertarget{ref-gervais_imaginaire_2016}{}}%
Gervais, Bertrand. 2016. {«~{Imaginaire de la fin du livre : figures du
livre et pratiques illitt{é}raires}~»}. \emph{LHT Fabula}, janvier.

\leavevmode\vadjust pre{\hypertarget{ref-gilbert_regarder_2019}{}}%
Gilbert, Zanna, et Alt Går Bra. 2019. {«~{Regarder en arri{è}re pour
aller de l'avant : Un entretien avec le collectif Alt G{å}r Bra}~»}.
Traduit par Étienne Gomez. \emph{Perspective. Actualit{é} en histoire de
l'art}, nᵒ 2 (décembre). {Institut national d'histoire de l'art}:143‑62.
\url{https://doi.org/10.4000/perspective.15051}.

\leavevmode\vadjust pre{\hypertarget{ref-gitelman_scripts_1999}{}}%
Gitelman, Lisa. 1999. \emph{Scripts, {Grooves}, and {Writing Machines}:
{Representing Technology} in the {Edison Era}}. {Stanford}: {Stanford
University Press}.

\leavevmode\vadjust pre{\hypertarget{ref-gitelman_always_2006}{}}%
Gitelman, Lisa. 2006. \emph{Always {Already New}: {Media}, {History} and
the {Data} of {Culture}}. {Cambridge, Mass}: {MIT Press}.

\leavevmode\vadjust pre{\hypertarget{ref-gleize_francis_1986}{}}%
Gleize, Jean-Marie. 1986. \emph{Francis {Ponge}}. Cahier de l'{Herne}
51. {Paris}: {{É}ditions du Seuil}.

\leavevmode\vadjust pre{\hypertarget{ref-gleize_ponge_2004}{}}%
Gleize, Jean-Marie, éd. 2004. \emph{{Ponge, r{é}solument}}. \emph{Ponge,
r{é}solument}. {Signes}. {Lyon}: {ENS {É}ditions}.
\url{https://doi.org/10.4000/books.enseditions.34828}.

\leavevmode\vadjust pre{\hypertarget{ref-goldsmith_day_2003}{}}%
Goldsmith, Kenneth. 2003. \emph{Day}. {Geoffrey Young}.

\leavevmode\vadjust pre{\hypertarget{ref-goldsmith_uncreative_2011}{}}%
Goldsmith, Kenneth. 2011. \emph{Uncreative {Writing}: {Managing
Language} in the {Digital Age}}. {New York}: {Columbia University
Press}.

\leavevmode\vadjust pre{\hypertarget{ref-goldsmith_ecriture_2018}{}}%
Goldsmith, Kenneth. 2018. \emph{{L'{é}criture sans {é}criture: du
langage {à} l'{â}ge num{é}rique}}. Traduit par François Bon. {Jean
Bo{î}te {É}ditions}.

\leavevmode\vadjust pre{\hypertarget{ref-goodman_languages_1976}{}}%
Goodman, Nelson. 1976. \emph{Languages of {Art}: {An Approach} to a
{Theory} of {Symbols}}. {Hackett Publishing}.

\leavevmode\vadjust pre{\hypertarget{ref-goody_domestication_1977}{}}%
Goody, Jack. 1977. \emph{The {Domestication} of the {Savage Mind}}.
Themes in the Social Sciences. {Cambridge London New York {[}etc.{]}}:
{Cambridge university press}.

\leavevmode\vadjust pre{\hypertarget{ref-goody_raison_1986}{}}%
Goody, Jack. (1979) 1986. \emph{{La raison graphique : la domestication
de la pens{é}e sauvage}}. Traduit par Jean Bazin. {Le sens commun}.
{Paris}: {{É}ditions de Minuit}.

\leavevmode\vadjust pre{\hypertarget{ref-grusin_radical_2015}{}}%
Grusin, Richard. 2015. {«~Radical {Mediation}~»}. \emph{Critical
Inquiry} 42 (1). {The University of Chicago Press}:124‑48.
\url{https://doi.org/10.1086/682998}.

\leavevmode\vadjust pre{\hypertarget{ref-guez_mort_2017}{}}%
Guez, Emmanuel, et Frédérique Vargoz. 2017. {«~{La mort de l'auteur
selon Friedrich Kittler}~»}. \emph{Appareil}, nᵒ 19 (octobre). {Maison
des Sciences de l'Homme Paris Nord}.
\url{https://doi.org/10.4000/appareil.2561}.

\leavevmode\vadjust pre{\hypertarget{ref-gunder_forming_2001}{}}%
Gunder, Anna. 2001. {«~Forming the {Text}, {Performing} the {Work} -
{Aspects} of {Media}, {Navigation}, and {Linking}~»}. \emph{Human IT:
Journal for Information Technology Studies as a Human Science} 5 (2-3).

\leavevmode\vadjust pre{\hypertarget{ref-haraway_manifeste_2007}{}}%
Haraway, Donna Jeanne. 2007. \emph{{Manifeste cyborg et autres essais :
sciences, fictions, f{é}minismes}}. Édité par Laurence Allard et
Delphine Gardey. {Essais}. {Paris}: {Exils}.

\leavevmode\vadjust pre{\hypertarget{ref-hayles_how_1999}{}}%
Hayles, N. Katherine. 1999. \emph{How {We Became Posthuman}: {Virtual
Bodies} in {Cybernetics}, {Literature}, and {Informatics}}. {Chicago,
Ill}: {University of Chicago Press}.

\leavevmode\vadjust pre{\hypertarget{ref-hayles_writing_2002}{}}%
Hayles, N. Katherine. 2002. \emph{Writing {Machines}}. A {Mediawork}
Pamphlet. {Cambridge}: {Mediawork}.

\leavevmode\vadjust pre{\hypertarget{ref-hayles_my_2005}{}}%
Hayles, N. Katherine. 2005. \emph{My {Mother} Was a {Computer}: {Digital
Subjects} and {Literary Texts}}. {Chicago}: {University of Chicago
Press}.

\leavevmode\vadjust pre{\hypertarget{ref-hayles_how_2012}{}}%
Hayles, N. Katherine. 2012. \emph{How {We Think}: {Digital Media} and
{Contemporary Technogenesis}}. {Chicago, IL}: {University of Chicago
Press}.

\leavevmode\vadjust pre{\hypertarget{ref-heidegger_question_1977}{}}%
Heidegger, Martin. 1977. \emph{The {Question Concerning Technology}, and
{Other Essays}}. {New York}: {Garland Pub}.

\leavevmode\vadjust pre{\hypertarget{ref-heidegger_parmenide_2011}{}}%
Heidegger, Martin. 2011. \emph{{Parm{é}nide: cours de Fribourg du
semestre d'hiver 1942 - 1943}}. Édité par Thomas Piel et Manfred S.
Frings. {Oeuvres de Martin Heidegger}. {Paris}: {Gallimard}.

\leavevmode\vadjust pre{\hypertarget{ref-heilbroner_machines_1967}{}}%
Heilbroner, Robert L. 1967. {«~Do {Machines Make History}?~»}
\emph{Technology and Culture} 8 (3):335.
\url{https://doi.org/10.2307/3101719}.

\leavevmode\vadjust pre{\hypertarget{ref-herbertz_zur_1909}{}}%
Herbertz, Richard. 1909. \emph{{Zur Psychologie des
Maschinenschreibens}}. {J. A. Barth}.

\leavevmode\vadjust pre{\hypertarget{ref-herrenschmidt_les_2007}{}}%
Herrenschmidt, Clarisse. 2007. \emph{Les Trois {É}critures: {Langue},
Nombre, Code}. Biblioth{è}que Des Sciences Humaines. {Paris, France}:
{Gallimard}.

\leavevmode\vadjust pre{\hypertarget{ref-hodges_alan_1992}{}}%
Hodges, Andrew. 1992. \emph{Alan {Turing}: {The Enigma}}. {London}:
{Vintage Books}.

\leavevmode\vadjust pre{\hypertarget{ref-hookway_interface_2014}{}}%
Hookway, Branden. 2014. \emph{Interface}. {Cambridge, Massachusetts}:
{The MIT Press}.

\leavevmode\vadjust pre{\hypertarget{ref-huang_henri_2014}{}}%
Huang, Bei. 2014. {«~{Henri Michaux et l'aventure du geste}~»}.
\emph{Litt{é}rature} 175 (3). {Paris}: {Armand Colin}:106‑22.
\url{https://doi.org/10.3917/litt.175.0106}.

\leavevmode\vadjust pre{\hypertarget{ref-hugo_oeuvres_2013}{}}%
Hugo, Victor. 2013. \emph{{{Œ}uvres compl{è}tes. Tome VII}}. Hachette.
{HACHETTBNF}.

\leavevmode\vadjust pre{\hypertarget{ref-husserl_ideas_2014}{}}%
Husserl, Edmund. 2014. \emph{Ideas for a {Pure Phenomenology} and
{Phenomenological Philosophy}. {First} Book: {General Introduction} to
{Pure Phenomenology}}. {Indianapolis/Cambridge}: {Hackett Publishing
Company}.

\leavevmode\vadjust pre{\hypertarget{ref-hyland_forgotten_2016}{}}%
Hyland, J. M. E. 2016. {«~The {Forgotten Turing}~»}. In \emph{The {Once}
and {Future Turing}: {Computing} the {World}}, édité par Andrew Hodges
et S. Barry Cooper, 20‑33. {Cambridge}: {Cambridge University Press}.
\url{https://doi.org/10.1017/CBO9780511863196.005}.

\leavevmode\vadjust pre{\hypertarget{ref-ian_daffern_skin_2012}{}}%
Ian Daffern. 2012. {«~Skin ({A Mortal Work} of {Art}) by {Shelley
Jackson}~»}.

\leavevmode\vadjust pre{\hypertarget{ref-ina_livre_1964}{}}%
INA. 1964. {«~{Le livre de poche et le m{é}pris}~»}.

\leavevmode\vadjust pre{\hypertarget{ref-ingarden_loeuvre_1983}{}}%
Ingarden, Roman. 1983. \emph{L'{œ}uvre d'art Litt{é}raire}. {L'{â}ge
d'homme}.

\leavevmode\vadjust pre{\hypertarget{ref-ingold_tools_1989}{}}%
Ingold, Tim. 1989. {«~Tools, {Minds} and {Machines} : An {Excursion} in
the {Philosophy} of {Technology}~»}. \emph{Techniques \& Culture. Revue
Semestrielle d'anthropologie Des Techniques}, nᵒ 12 (juin). {Les
{É}ditions de l'EHESS}. \url{https://doi.org/10.4000/tc.805}.

\leavevmode\vadjust pre{\hypertarget{ref-ingold_perception_2000}{}}%
Ingold, Tim. 2000. \emph{The {Perception} of the {Environment}: {Essays}
on {Livelihood}, {Dwelling} and {Skill}}. {London}: {Routledge}.
\url{https://doi.org/10.4324/9780203466025}.

\leavevmode\vadjust pre{\hypertarget{ref-ingold_outil_2010}{}}%
Ingold, Tim. 2010. {«~{L'Outil, l'esprit et la machine : Une excursion
dans la philosophie de la "technologie"}~»}. Traduit par Arundhati
Virmani. \emph{Techniques \& Culture. Revue semestrielle d'anthropologie
des techniques}, nᵒ 54-55 (juin). {Les {É}ditions de l'EHESS}:291‑311.
\url{https://doi.org/10.4000/tc.5004}.

\leavevmode\vadjust pre{\hypertarget{ref-ingold_making_2013}{}}%
Ingold, Tim. 2013. \emph{Making: {Anthropology}, {Archaeology}, {Art}
and {Architecture}}. {Milton Park, Abingdon, Oxon}: {Routledge}.

\leavevmode\vadjust pre{\hypertarget{ref-ingold_lines_2016}{}}%
Ingold, Tim. 2016. \emph{Lines: {A Brief History}}. Routledge
{Classics}. {London New York}: {Routledge}.

\leavevmode\vadjust pre{\hypertarget{ref-ingold_faire_2017}{}}%
Ingold, Tim, Hervé Gosselin, et Hicham-Stéphane Afeissa. 2017.
\emph{{Faire: anthropologie, arch{é}ologie, art et architecture}}.
{Bellevaux}: {{É}ditions Dehors}.

\leavevmode\vadjust pre{\hypertarget{ref-ingold_entre_2021}{}}%
Ingold, Tim, et La manufacture des idées. 2021. {«~{Entre les
lignes}~»}.

\leavevmode\vadjust pre{\hypertarget{ref-jackson_author_2003}{}}%
Jackson, Shelley. 2003. {«~Author {Announces Mortal Work} of {Art}~»}.
\emph{Ineradicable Stain}. https://ineradicablestain.com/skin-call.html.

\leavevmode\vadjust pre{\hypertarget{ref-jean_a_baudot_machine_1964}{}}%
Jean A. Baudot. 1964. \emph{{La machine {à} {é}crire: mise en marche et
programm{é}e par Jean A. Baudot}}. {{É}ditions du Jour}.

\leavevmode\vadjust pre{\hypertarget{ref-jeanneret_lenonciation_2005}{}}%
Jeanneret, Yves, et Emmanuël Souchier. 2005. {«~{L'{é}nonciation
{é}ditoriale dans les {é}crits d'{é}cran}~»}. \emph{Communication \&
Langages} 145 (1). {Armand Colin}:3‑15.
\url{https://doi.org/10.3406/colan.2005.3351}.

\leavevmode\vadjust pre{\hypertarget{ref-johnson_bookrolls_2003}{}}%
Johnson, William A. 2003. \emph{Bookrolls and {Scribes} in
{Oxyrhynchus}}. {University of Toronto Press}.
\url{https://doi.org/10.3138/9781442671515}.

\leavevmode\vadjust pre{\hypertarget{ref-johnston_1964_2008}{}}%
Johnston, Dave. 2008. {«~1964: {Baudot}, {La} Machine {à} {É}crire
{\textendash} {Digital Poetics Prehistoric}~»}. \emph{Digital Poetics
Prehistoric}.

\leavevmode\vadjust pre{\hypertarget{ref-junger_cabane_2014}{}}%
Jünger, Ernst. 2014. \emph{{La cabane dans la vigne: Journal
1945-1948}}. Édité par Julien Hervier. Traduit par Maurice Betz.
{Titres}, titre 175. {Paris}: {C. Bourgois}.

\leavevmode\vadjust pre{\hypertarget{ref-junod_transparence_2004}{}}%
Junod, Philippe. 2004. \emph{{Transparence et opacit{é}: Essai sur les
fondements th{é}oriques de l'art moderne pour une nouvelle lecture de
Konrad Fiedler}}. {Rayon art}. {N{î}mes}: {J. Chambon}.

\leavevmode\vadjust pre{\hypertarget{ref-kapelusz_dispositifs_2016}{}}%
Kapelusz, Anyssa. 2016. {«~Dispositifs Sonores et {É}coute Performative
: Le Cas de l'{Autoteatro} Par La Compagnie {Rotozaza}~»}.
\emph{L'Annuaire th{é}{â}tral}, nᵒ 56-57 (août):149‑59.
\url{https://doi.org/10.7202/1037335ar}.

\leavevmode\vadjust pre{\hypertarget{ref-chapple_theatre_2006}{}}%
Kattenbelt, Chiel. 2006. {«~Theatre as the {Art} of the {Performer} and
the {Stage} of {Intermediality}~»}. In \emph{Intermediality in {Theatre}
and {Performance}}. {BRILL}.
\url{https://doi.org/10.1163/9789401210089}.

\leavevmode\vadjust pre{\hypertarget{ref-kattenbelt2015intermedialite}{}}%
Kattenbelt, Chiel. 2015. {«~L'interm{é}dialit{é} Comme Mode de
Performativit{é}~»}. \emph{Th{é}{â}tre et interm{é}dialit{é}, Villeneuve
d'Ascq, Presses universitaires du Septentrion}, 101‑15.

\leavevmode\vadjust pre{\hypertarget{ref-kayser_deconstruction_2019}{}}%
Kayser, Cédric. 2019. {«~{D{é}construction et spatialit{é} dans
Patchwork Girl de Shelley Jackson}~»}. \emph{Sens public}, avril.

\leavevmode\vadjust pre{\hypertarget{ref-kenyon_books_1932}{}}%
Kenyon, Frederick. 1932. \emph{Books {And Readers In Ancient Greece And
Rome}}. {Oxford University Press}.

\leavevmode\vadjust pre{\hypertarget{ref-kirschenbaum_extreme_2016}{}}%
Kirschenbaum, Matthew. 2016. {«~Extreme {Inscription}: {A Grammatology}
of the {Hard Drive}~»}. In \emph{New {Media}, {Old Media}: {A History}
and {Theory Reader}}, Second edition. {New York, NY}: {Routledge}.

\leavevmode\vadjust pre{\hypertarget{ref-kittler_austreibung_1980}{}}%
Kittler, Friedrich A. 1980. \emph{{Austreibung des Geistes aus den
Geisteswissenschaften Programme des Poststrukturalismus}}. {M{ü}nchen ;
Wien ; Z{ü}rich}: {Paderborn,}.

\leavevmode\vadjust pre{\hypertarget{ref-kittler_aufschreibesysteme_1985}{}}%
Kittler, Friedrich A. 1985. \emph{{Aufschreibesysteme 1800/1900}}. 3.,
vollst. {ü}berarb. Neuaufl. {M{ü}nchen}: {Fink}.

\leavevmode\vadjust pre{\hypertarget{ref-kittler_grammophon_1986}{}}%
Kittler, Friedrich A. 1986. \emph{Grammophon {Film Typewriter}}.
{Berlin}: {Brinkmann \& Bose}.

\leavevmode\vadjust pre{\hypertarget{ref-kittler_discourse_1990}{}}%
Kittler, Friedrich A. 1990. \emph{Discourse {Networks} 1800/1900}.
{Stanford, Calif}: {Stanford University Press}.

\leavevmode\vadjust pre{\hypertarget{ref-kittler_draculas_1993}{}}%
Kittler, Friedrich A. 1993. \emph{{Draculas Verm{ä}chtnis: technische
Schriften}}. 1. Aufl. {Reclam-Bibliothek} 1476. {Leipzig}: {Reclam}.

\leavevmode\vadjust pre{\hypertarget{ref-kittler_gramophone_1999}{}}%
Kittler, Friedrich A. 1999. \emph{Gramophone, {Film}, {Typewriter}}.
Writing {Science}. {Stanford, Calif}: {Stanford University Press}.

\leavevmode\vadjust pre{\hypertarget{ref-kittler_geschichte_2004}{}}%
Kittler, Friedrich A. 2004. {«~Geschichte Der
{Kommunikationstechniken}~»}. In \emph{Semiotik}, édité par Roland
Posner, Klaus Robering, et Thomas A. Sebeok, 3345‑57. {Walter de
Gruyter}. \url{https://doi.org/10.1515/9783110179620.4.15.3345}.

\leavevmode\vadjust pre{\hypertarget{ref-kittler_mode_2015}{}}%
Kittler, Friedrich A. 2015. \emph{{Mode prot{é}g{é}}}. Édité par
Emmanuel Guez. Traduit par Frédérique Vargoz. {La petite collection
Arts-H2H}. {Dijon Saint-Denis}: {les Presses du r{é}el Labex Arts-H2H}.

\leavevmode\vadjust pre{\hypertarget{ref-kittler_exorciser_2017}{}}%
Kittler, Friedrich A. 2017. {«~{Exorciser l'homme des sciences humaines
: programmes du poststructuralisme}~»}. Traduit par Slaven Waelti.
\emph{Appareil}, nᵒ 19 (octobre). {Maison des Sciences de l'Homme Paris
Nord}. \url{https://doi.org/10.4000/appareil.2522}.

\leavevmode\vadjust pre{\hypertarget{ref-kittler_gramophone_2018}{}}%
Kittler, Friedrich A. 2018. \emph{{Gramophone, film, typewriter}}. Édité
par Emmanuel Alloa et Emmanuel Guez. Traduit par Frédérique Vargoz.
{M{é}dias / th{é}ories}. {Dijon}: {les Presses du r{é}el}.

\leavevmode\vadjust pre{\hypertarget{ref-klock-fontanille_supports_2010}{}}%
Klock-Fontanille, Isabelle. 2010. {«~{Des supports pour {é}crire d'Uruk
{à} Internet}~»}. \emph{Le fran{ç}ais aujourd'hui} 170 (3). {Paris}:
{Armand Colin}:13‑30. \url{https://doi.org/10.3917/lfa.170.0013}.

\leavevmode\vadjust pre{\hypertarget{ref-koltes_dans_1986}{}}%
Koltès, Bernard-Marie. 1986. \emph{Dans La Solitude Des Champs de
Coton}. {Paris}: {Editions de Minuit}.

\leavevmode\vadjust pre{\hypertarget{ref-kramer_medium_2015}{}}%
Krämer, Sybille. 2015. \emph{Medium, {Messenger}, {Transmission}: An
{Approach} to {Media Philosophy}}. Recursions: Theories of Media,
Materiality, Et Cultural Techniques. {Amsterdam}: {Amsterdam university
press}.

\leavevmode\vadjust pre{\hypertarget{ref-krauss_optical_1994}{}}%
Krauss, Rosalind E. 1994. \emph{Optical {Unconscious}}. 1st paperback
ed. An {October} Book. {Cambridge, Mass.}: {MIT Press}.

\leavevmode\vadjust pre{\hypertarget{ref-kristeva_pillow_1994}{}}%
Kristeva, Tzvetana. 1994. {«~The {Pillow Hook}~»}. \emph{Nichibunken
Japan review : bulletin of the International Research Center for
Japanese Studies} 5:15‑54.

\leavevmode\vadjust pre{\hypertarget{ref-lacan_ecrits_1966}{}}%
Lacan, Jacques. 1966. \emph{{É}crits}. Le {Champ} Freudien. {Paris}:
{{É}ditions du Seuil}.

\leavevmode\vadjust pre{\hypertarget{ref-lange-berndt_materiality_2015}{}}%
Lange-Berndt, Petra, éd. 2015. \emph{Materiality}. Documents of
Contemporary Art. {London : Cambridge, Massachusetts}: {Whitechapel
Gallery ; The MIT Press}.

\leavevmode\vadjust pre{\hypertarget{ref-lanham_electronic_1995}{}}%
Lanham, Richard A. 1995. \emph{The {Electronic Word}: {Democracy},
{Technology}, and the {Arts}}. {Chicago, IL}: {University of Chicago
Press}.

\leavevmode\vadjust pre{\hypertarget{ref-lapprand_poetique_1998}{}}%
Lapprand, Marc. 1998. \emph{{Po{é}tique de l'Oulipo}}. {Faux titre} 142.
{Amsterdam Atlanta (Ga.)}: {Rodopi}.

\leavevmode\vadjust pre{\hypertarget{ref-larrue_du_2016}{}}%
Larrue, Jean-Marc. 2016a. {«~{Du m{é}dia {à} la m{é}diation~: les trente
ans de la pens{é}e interm{é}diale et la r{é}sistance th{é}{â}trale}~»}.
In \emph{{Th{é}{â}tre et interm{é}dialit{é}}}, 27‑56. {Arts du spectacle
{\textendash} Images et sons}. {Villeneuve d'Ascq}: {Presses
universitaires du Septentrion}.
\url{https://doi.org/10.4000/books.septentrion.8158}.

\leavevmode\vadjust pre{\hypertarget{ref-larrue_theatre_2016}{}}%
Larrue, Jean-Marc, éd. 2016b. \emph{{Th{é}{â}tre et
interm{é}dialit{é}}}. \emph{Th{é}{â}tre et interm{é}dialit{é}}. {Arts du
spectacle {\textendash} Images et sons}. {Villeneuve d'Ascq}: {Presses
universitaires du Septentrion}.
\url{https://doi.org/10.4000/books.septentrion.8153}.

\leavevmode\vadjust pre{\hypertarget{ref-larrue_intermedialite_2020}{}}%
Larrue, Jean-Marc. 2020. {«~{De l'interm{é}dialit{é} {à}
l'excommunication}~»}. \emph{Cahiers d'{É}tudes Germaniques} 1 volume
(79). {Universit{é} Aix-Marseille (AMU)}:31‑48.
\url{https://doi.org/10.4000/ceg.12226}.

\leavevmode\vadjust pre{\hypertarget{ref-larrue_media_2019}{}}%
Larrue, Jean-Marc, et Marcello Vitali-Rosati. 2019. \emph{Media {Do Not
Exist}: {Performativity} and {Mediating Conjunctures}}. {Institute of
Network Cultures}.

\leavevmode\vadjust pre{\hypertarget{ref-lassegue_turing_1998}{}}%
Lassègue, Jean. 1998. \emph{Turing}. Figures Du Savoir 12. {Paris}:
{Belles lettres}.

\leavevmode\vadjust pre{\hypertarget{ref-latour_agency_2014}{}}%
Latour, Bruno. 2014. {«~Agency at the {Time} of the {Anthropocene}~»}.
\emph{New Literary History} 45 (1). {The Johns Hopkins University
Press}:1‑18. \url{https://www.jstor.org/stable/24542578}.

\leavevmode\vadjust pre{\hypertarget{ref-laurens_les_2016}{}}%
Laurens, Pierre. 2016. \emph{{Les mots latins pour Mathilde : petites
le{ç}ons d'une grande langue}}. {Paris}: {les Belles lettres}.

\leavevmode\vadjust pre{\hypertarget{ref-36920}{}}%
Le Lionnais, François. 1973. {«~La {LiPo}. {Le} Premier Manifeste~»}.
Papier. In \emph{Oulipo, La Litt{é}rature Potentielle (Cr{é}ations
Re-Cr{é}ations R{é}cr{é}ations)}, 19‑21. {Paris}: {Gallimard}.

\leavevmode\vadjust pre{\hypertarget{ref-le_tellier_esthetique_2006}{}}%
Le Tellier, Hervé. 2006. \emph{{Esth{é}tique de l'OULIPO}}. {B{è}gles}:
{le Castor astral}.

\leavevmode\vadjust pre{\hypertarget{ref-leiris_glossaire_2014}{}}%
Leiris, Michel. 2014. \emph{{Glossaire j'y serre mes gloses suivi de
Bagatelles v{é}g{é}tales}}. Édité par André Masson et Louis Yvert.
{Collection po{é}sie} 492. {Paris}: {Gallimard}.

\leavevmode\vadjust pre{\hypertarget{ref-leroi-gourhan_geste_2014}{}}%
Leroi-Gourhan, André. (1964) 2014. \emph{{Le geste et la parole.
{[}I{]}, Technique et langage}}. {Paris}: {Albin Michel}.

\leavevmode\vadjust pre{\hypertarget{ref-lessing_laocoon_2011}{}}%
Lessing, Gotthold Ephraim, et Frédéric Teinturier. (1766) 2011.
\emph{{Laocoon ou Des fronti{è}res respectives de la peinture et de la
po{é}sie}}. {L'esprit et les formes} 35. {Paris}: {Klincksieck}.

\leavevmode\vadjust pre{\hypertarget{ref-levin_becker_many_2012}{}}%
Levin Becker, Daniel. 2012. \emph{Many {Subtle Channels}: {In Praise} of
{Potential Literature}}. {Cambridge, Mass}: {Harvard University Press}.

\leavevmode\vadjust pre{\hypertarget{ref-levi-strauss_pensee_2010}{}}%
Lévi-Strauss, Claude. (1962) 2010. \emph{{La pens{é}e sauvage}}. {Agora}
2. {Paris}: {Presses Pocket}.

\leavevmode\vadjust pre{\hypertarget{ref-levy_intelligence_1997}{}}%
Lévy, Pierre. 1997. \emph{L'intelligence Collective : Pour Une
Anthropologie Du Cyberspace}. La {D{é}couverte}/{Poche} 27. {Paris}: {La
D{é}couverte}.

\leavevmode\vadjust pre{\hypertarget{ref-lichtenthal_ii._2018}{}}%
Lichtenthal, Julia. 2018. {«~L'invention de La Topographie Dans {Un}
Coup de D{é}s Jamais n'abolira Le Hasard:~»} In \emph{Europe En
Mouvement 2}, 39‑46. {Hermann}.
\url{https://doi.org/10.3917/herm.heurg.2018.03.0039}.

\leavevmode\vadjust pre{\hypertarget{ref-locke_essay_1997}{}}%
Locke, John. 1997. \emph{An {Essay Concerning Human Understanding}}.
Édité par R. S. Woolhouse. {London}: {New York : Penguin Books}.

\leavevmode\vadjust pre{\hypertarget{ref-lorusso_liquider_2022}{}}%
Lorusso, Silvio. 2022. {«~{Liquider l'utilisateur}~»}. Traduit par
Sophie Garnier. \emph{T{è}que} 1 (1). {Audimat {É}ditions}:10‑57.
\url{https://doi.org/10.3917/tequ.001.0010}.

\leavevmode\vadjust pre{\hypertarget{ref-lucretius_caroli_1850}{}}%
Lucretius, Carus. 1850. \emph{{Caroli Lachmanni in T. Lucretii Cari De
rerum natura libros commentarius}}. Édité par Karl Lachmann. {Berolini :
Impensis G. Reimeri}.

\leavevmode\vadjust pre{\hypertarget{ref-ludovico_post-digital_2012}{}}%
Ludovico, Alessandro. 2012. \emph{Post-{Digital Print}: {The Mutation}
of {Publishing Since} 1894}. Onomatopee 77. {Eindhoven}: {Onomatopee}.

\leavevmode\vadjust pre{\hypertarget{ref-macdonald_night_2015}{}}%
MacDonald, Tanis. 2015. {«~Night in a {Box}: {Anne Carson}'s {Nox} and
the {Materiality} of {Elegy}~»}. In \emph{Material {Cultures} in
{Canada}}, 51‑64. {Wilfried Laurier UP}.

\leavevmode\vadjust pre{\hypertarget{ref-mak_how_2011}{}}%
Mak, Bonnie. 2011. \emph{How the {Page Matters}}. Studies in Book et
Print Culture Series. {Toronto ; Buffalo}: {University of Toronto
Press}.

\leavevmode\vadjust pre{\hypertarget{ref-mallarme_jamais_1897}{}}%
Mallarmé, Stéphane. 1897a. \emph{{Jamais un coup de d{é}s n'abolira le
hasard : {É}preuves d'imprimerie}}. {Paris}: {A. Vollard}.

\leavevmode\vadjust pre{\hypertarget{ref-mallarme_coup_1897}{}}%
Mallarmé, Stéphane. 1897b. {«~Un {Coup} de {D{é}} Jamais n'abolira Le
{Hasard}~»}. \emph{Cosmopolis}, nᵒ 17.

\leavevmode\vadjust pre{\hypertarget{ref-mallarme_correspondance_1969}{}}%
Mallarmé, Stéphane. (1886-1889) 1969. {«~{À} {Edmond Deman}, 28 Avril
1888~»}. In \emph{Correspondance}, édité par Henri Mondor et Lloyd James
Austen, 3:188. {NRF}. {Paris}: {Gallimard}.

\leavevmode\vadjust pre{\hypertarget{ref-ronat_coup_1981}{}}%
Mallarmé, Stéphane. 1981. \emph{Un Coup de D{è}s Jamais n'abolira Le
Hasard}. Édité par Tipor Papp et Mitsou Ronat. {Paris}: {Change errant}.

\leavevmode\vadjust pre{\hypertarget{ref-mallarme_ux153uvres_1998}{}}%
Mallarmé, Stéphane. 1998. \emph{{{Œ}uvres compl{è}tes}}. Édité par
Bertrand Marchal. {Biblioth{è}que de la Pl{é}iade}. {Paris}:
{Gallimard}.

\leavevmode\vadjust pre{\hypertarget{ref-mallarme_correspondance_1999}{}}%
Mallarmé, Stéphane. 1999. \emph{{Correspondance compl{è}te 1862 -
1871}}. Édité par Bertrand Marchal et Yves Bonnefoy. {Collection Folio
Classique} 2678. {Paris}: {Gallimard}.

\leavevmode\vadjust pre{\hypertarget{ref-mallarme_quant_2004}{}}%
Mallarmé, Stéphane. 2004a. \emph{{Quant au livre}}. {Tours Paris}:
{Farrago L. Scheer}.

\leavevmode\vadjust pre{\hypertarget{ref-mallarme_coup_2004}{}}%
Mallarmé, Stéphane. 2004b. \emph{{Un coup de d{é}s jamais n'abolira le
hasard}}. {Paris}: {Michel Pierson \& Ptyx, {É}diteurs}.

\leavevmode\vadjust pre{\hypertarget{ref-mallarme_divagations_2010}{}}%
Mallarmé, Stéphane. 2010. \emph{{Divagations}}. {Montpellier}:
{publie.net}.

\leavevmode\vadjust pre{\hypertarget{ref-mallarme_coup_2014}{}}%
Mallarmé, Stéphane. 2014. \emph{{Un coup de d{é}s jamais n'abolira le
hasard}}. Faksimile-Edition der Ausgabe 1914. {Paris}: {Gallimard}.

\leavevmode\vadjust pre{\hypertarget{ref-mallock_human_2005}{}}%
Mallock, William H. 2005. \emph{A {Human Document}}. {United States}:
{Elibron Classics : Adamant Media Corporation}.

\leavevmode\vadjust pre{\hypertarget{ref-malloy_cybertext_1998}{}}%
Malloy, Judy, et Espen J. Aarseth. 1998. {«~Cybertext, {Perspectives} on
{Ergodic Literature}~»}. \emph{Leonardo Music Journal} 8:77.
\url{https://doi.org/10.2307/1513408}.

\leavevmode\vadjust pre{\hypertarget{ref-manning_for_2020}{}}%
Manning, Erin. 2020. \emph{For a {Pragmatics} of the {Useless}}. Thought
in the Act. {Durham}: {Duke University Press}.

\leavevmode\vadjust pre{\hypertarget{ref-manovich_language_2001}{}}%
Manovich, Lev. 2001. \emph{The {Language} of {New Media}}. Leonardo.
{Cambridge, Mass.}: {MIT Press}.

\leavevmode\vadjust pre{\hypertarget{ref-manovitch_langage_2010}{}}%
Manovitch, Lev. 2010. \emph{{Le langage des nouveaux m{é}dias}}. Traduit
par Richard Crevier. {Perceptions}. {Dijon}: {les Presses du r{é}el}.

\leavevmode\vadjust pre{\hypertarget{ref-marinetti_i_1925}{}}%
Marinetti, Filippo Tommaso. 1925. \emph{I Nuovi Poeti Futuristi}.
Edizioni Futuriste di "Poesia". {Roma}.

\leavevmode\vadjust pre{\hypertarget{ref-mariniello_2007}{}}%
Mariniello, Silvestra. 2007. \emph{{Appareil et interm{é}dialit{é}}}.
{Esth{é}tiques}. {Paris}: {Harmattan}.

\leavevmode\vadjust pre{\hypertarget{ref-martin_lorsque_2021}{}}%
Martin, Côme. 2021. {«~{Lorsque l'objet devient virtuel}~»}. \emph{Sens
public}, nᵒ SP1486 (juin). {D{é}partement des litt{é}ratures de langue
fran{ç}aise}.

\leavevmode\vadjust pre{\hypertarget{ref-martinuik_careful_2011}{}}%
Martinuik, Lorraine. 2011. {«~A {Careful Assemblage}~»}. \emph{Jacket2}
8 (décembre).

\leavevmode\vadjust pre{\hypertarget{ref-marx_capital_2014}{}}%
Marx, Karl. 2014. \emph{{Le capital : Critique de l'{é}conomie
politique}}. Édité par Jean-Pierre Lefebvre. Nouvelle {é}d. {Quadrige}.
{Paris}: {PUF}.

\leavevmode\vadjust pre{\hypertarget{ref-masure_vivre_2019}{}}%
Masure, Anthony. 2019. {«~{Vivre dans les programmes}~»}.
\emph{Multitudes} 74 (1). {Paris}: {Association Multitudes}:176‑81.
\url{https://doi.org/10.3917/mult.074.0176}.

\leavevmode\vadjust pre{\hypertarget{ref-maxwell_mind_2019}{}}%
Maxwell, John, Erik Hanson, Leena Desai, Carmen Tiampo, Kim O'Donnell,
Avvai Ketheeswaran, Melody Sun, Emma Walter, et Ellen Michelle. 2019.
\emph{Mind the {Gap}: {A Landscape Analysis} of {Open Source Publishing
Tools} and {Platforms}}. 1ʳᵉ éd. {PubPub}.
\url{https://doi.org/10.21428/6bc8b38c.2e2f6c3f}.

\leavevmode\vadjust pre{\hypertarget{ref-mayaux_dessins_2022}{}}%
Mayaux, Catherine. 2022. {«~Dessins Comment{é}s Ou Le Fant{ô}me Du
Po{è}te~»}. In \emph{Henri {Michaux} : {Corps} et Savoir}, édité par
Pierre Grouix et Jean-Michel Maulpoix, 17‑31. Signes. {Lyon}: {ENS
{É}ditions}.

\leavevmode\vadjust pre{\hypertarget{ref-mccarty_modeling_2004}{}}%
McCarty, Willard. 2004. {«~Modeling: {A Study} in {Words} and
{Meanings}~»}. In \emph{A {Companion} to {Digital Humanities}}, 254‑70.
{John Wiley \& Sons, Ltd}.
\url{https://doi.org/10.1002/9780470999875.ch19}.

\leavevmode\vadjust pre{\hypertarget{ref-mcluhan_understanding_1964}{}}%
McLuhan, Marshall. 1964. \emph{Understanding {Media}: {The Extensions}
of {Man}}. {New-York}: {McGraw-Hill}.

\leavevmode\vadjust pre{\hypertarget{ref-mcluhan_pour_1977}{}}%
McLuhan, Marshall. 1977. \emph{{Pour comprendre les m{é}dia : les
prolongements technologiques de l'homme}}. Traduit par Jean Paré.
{Collection Points} 83. {Paris}: {{É}ditions du Seuil}.

\leavevmode\vadjust pre{\hypertarget{ref-mcluhan_understanding_1994}{}}%
McLuhan, Marshall. 1994. \emph{Understanding {Media}: {The Extensions}
of {Man}}. 1st MIT Press ed. {Cambridge, Mass}: {MIT Press}.

\leavevmode\vadjust pre{\hypertarget{ref-mcpherson_feminist_2018}{}}%
McPherson, Tara. 2018. \emph{Feminist in a {Software Lab}: {Difference}
+ {Design}}. {MetaLABprojects}. {Cambridge, Massachusetts ; London,
England}: {Harvard University Press}.

\leavevmode\vadjust pre{\hypertarget{ref-mechoulan_culture_2008}{}}%
Méchoulan, Eric. 2008. \emph{{La culture de la m{é}moire : ou comment se
d{é}barrasser du pass{é} ?}} {Champ libre}. {Montr{é}al}: {Presses de
l'Universit{é} de Montr{é}al}.

\leavevmode\vadjust pre{\hypertarget{ref-mechoulan_dou_2010}{}}%
Méchoulan, Eric. 2010. \emph{D'o{ù} Nous Viennent Nos Id{é}es ?
{M{é}taphysique} et Interm{é}dialit{é}}. Le {Soi} et l'{Autre}.
{Montr{é}al}: {VBL {É}diteur}.

\leavevmode\vadjust pre{\hypertarget{ref-mechoulan_intermedialite_2017}{}}%
Méchoulan, Eric. 2017. {«~{Interm{é}dialit{é}, ou comment penser les
transmissions}~»}. \emph{Fabula Colloques}, mars.

\leavevmode\vadjust pre{\hypertarget{ref-mellet_penser_2020}{}}%
Mellet, Margot. 2020. {«~Penser Le Palimpseste Num{é}rique. {Le} Projet
d'{é}dition Num{é}rique Collaborative de l'{Anthologie} Palatine~»}.
\emph{Captures}, Hors-Dossier, 5 (1).

\leavevmode\vadjust pre{\hypertarget{ref-mellet_defaire_2021}{}}%
Mellet, Margot. 2021a. {«~{D{é}faire et rem{é}dier le livre. Analyse de
la figure-{é}cran du livre dans The Pillow Book de Peter Greenaway}~»}.
\emph{{É}tudes du livre au XXIe si{è}cle}.

\leavevmode\vadjust pre{\hypertarget{ref-mellet_manifeste_2021}{}}%
Mellet, Margot. 2021b. {«~{Manifeste des petites mains}~»}.
\emph{Blank.blue}.

\leavevmode\vadjust pre{\hypertarget{ref-mellet_celles_2022}{}}%
Mellet, Margot. 2022. {«~{Celles qui survivent aux Hommes}~»}.

\leavevmode\vadjust pre{\hypertarget{ref-mellet__2023}{}}%
Mellet, Margot. 2023.{«~{{\guillemotleft}Mais {é}taient-ce des signes ?
{\guillemotright}}~»}. \emph{Revue F{é}mur} 7 (juin).

\leavevmode\vadjust pre{\hypertarget{ref-mellet_image_2024}{}}%
Mellet, Margot. 2024a. {«~L'image Organique Du Texte Num{é}rique~»}. In
\emph{La Pens{é}e de l'organe}. {Presses universitaires de
l'Universit{é} de Laval}.

\leavevmode\vadjust pre{\hypertarget{ref-mellet_poursuite_2024}{}}%
Mellet, Margot. 2024b. {«~La Poursuite Du Fait Litt{é}raire~»}.
\emph{Imaginations}.

\leavevmode\vadjust pre{\hypertarget{ref-mellet_passes_2024}{}}%
Mellet, Margot, et Mathilde Verstraete. 2024. {«~Pass{é}s et Pr{é}sents
Anthologiques. {Le} Projet d'{é}dition Num{é}rique Collaborative de
l'{Anthologie} Grecque~»}. In \emph{Communaut{é}s et Pratiques
d'{é}critures Des Patrimoines et Des M{é}moires}. {Paris}: {Presses
universitaires de Paris Nanterre}.

\leavevmode\vadjust pre{\hypertarget{ref-merzeau_du_2009}{}}%
Merzeau, Louise. 2009. {«~{Du signe {à} la trace : L'information sur
mesure}~»}. \emph{Hermes, La Revue} n{\textdegree} 53 (1):21‑29.

\leavevmode\vadjust pre{\hypertarget{ref-merzeau_editorialisation_2013}{}}%
Merzeau, Louise. 2013. {«~{É}ditorialisation Collaborative d'un
{É}v{é}nement : {L}'exemple Des {Entretiens} Du Nouveau Monde Industriel
2012~»}. \emph{Communication et organisation}, nᵒ 43 (juin):105‑22.
\url{https://doi.org/10.4000/communicationorganisation.4158}.

\leavevmode\vadjust pre{\hypertarget{ref-merzeau_entre_nodate}{}}%
Merzeau, Louise. 2014. {«~{Entre {é}v{è}nement et document : vers
l'environnement-support}~»}.

\leavevmode\vadjust pre{\hypertarget{ref-meunier_humanites_2014}{}}%
Meunier, Jean-Guy. 2014. {«~{Humanit{é}s num{é}riques ou
computationnelles : enjeux herm{é}neutiques}~»}. \emph{Sens public}.
{D{é}partement des litt{é}ratures de langue fran{ç}aise}.
\url{https://doi.org/10.7202/1043651ar}.

\leavevmode\vadjust pre{\hypertarget{ref-michaux_plume_1994}{}}%
Michaux, Henri. 1994. \emph{{Plume}}. Nouv. {é}d., revue et corr.
{Paris}: {Gallimard}.

\leavevmode\vadjust pre{\hypertarget{ref-michaux_nuit_1997}{}}%
Michaux, Henri. 1997a. \emph{{La nuit remue}}. Nouvelle {é}d. rev. et
corr. {Collection Po{é}sie} 217. {Paris}: {Gallimard}.

\leavevmode\vadjust pre{\hypertarget{ref-michaux_barbare_1997}{}}%
Michaux, Henri. 1997b. \emph{{Un barbare en Asie}}. Nouv. {é}d., revue
et corr. {Collection l'imaginaire} 164. {Paris}: {Gallimard}.

\leavevmode\vadjust pre{\hypertarget{ref-michaux_emergences_2000}{}}%
Michaux, Henri. (1972) 1998. \emph{{É}mergences - {R{é}surgences}}.
{Milano}: {Skira}.

\leavevmode\vadjust pre{\hypertarget{ref-michaux_oeuvres_1998}{}}%
Michaux, Henri. 1998. \emph{{Œ}uvres Compl{è}tes}. Édité par Raymond
Bellour et Ysé Tran. Biblioth{è}que de La {Pl{é}iade} 444, 475, 506.
{Paris}: {Gallimard}.

\leavevmode\vadjust pre{\hypertarget{ref-miller_cultural_2013}{}}%
Miller, Peter N., éd. 2013. \emph{Cultural {Histories} of the {Material
World}}. The {Bard Graduate Center Cultural Histories} of the {Material
World}. {Ann Arbor}: {The University of Michigan Press}.

\leavevmode\vadjust pre{\hypertarget{ref-mitcham_types_1978}{}}%
Mitcham, Carl. 1978. {«~Types of {Technology}~»}. \emph{Research in
Philosophy \& Technology} 1:229‑94.

\leavevmode\vadjust pre{\hypertarget{ref-mitcham_philosophy_1979}{}}%
Mitcham, Carl. 1979. {«~Philosophy and the {History} of {Technology}~»}.
\emph{The History and Philosophy of Technology}, janvier, 163‑201.

\leavevmode\vadjust pre{\hypertarget{ref-monjour_dibutade_2015}{}}%
Monjour, Servanne. 2015. {«~{Dibutade 2.0 : la "femme-auteur" {à}
l'{è}re du num{é}rique}~»}. In \emph{{Sens public}}. {Association Sens
public}.

\leavevmode\vadjust pre{\hypertarget{ref-monjour_remediation_2018}{}}%
Monjour, Servanne. 2018a. {«~{De la rem{é}diation {à} la
r{é}trom{é}diation}~»}. In \emph{{Mythologies postphotographiques.
L'invention litt{é}raire de l'image num{é}rique}}, 55‑62. {Parcours
Num{é}riques} 10. {Montr{é}al}: {Les Presses de l'Universit{é} de
Montr{é}al}.

\leavevmode\vadjust pre{\hypertarget{ref-monjour_mythologies_2018}{}}%
Monjour, Servanne. 2018b. \emph{{Mythologies postphotographiques.
L'invention litt{é}raire de l'image num{é}rique}}. {Parcours
num{é}riques} 10. {Montr{é}al}: {Les Presses de l'Universit{é} de
Montr{é}al}.

\leavevmode\vadjust pre{\hypertarget{ref-monjour_introduction_2018}{}}%
Monjour, Servanne. 2018c. {«~{Introduction}~»}. In \emph{{Mythologies
postphotographiques. L'invention litt{é}raire de l'image num{é}rique}}.
{Les Presses de l'Universit{é} de Montr{é}al}.

\leavevmode\vadjust pre{\hypertarget{ref-monjour_modeanamorphique_2018}{}}%
Monjour, Servanne. 2018d. {«~{Le mod{è}le anamorphique}~»}. In
\emph{{Mythologies postphotographiques. L'invention litt{é}raire de
l'image num{é}rique}}. {Les Presses de l'Universit{é} de Montr{é}al}.

\leavevmode\vadjust pre{\hypertarget{ref-monjour_litterature_2020}{}}%
Monjour, Servanne. 2020. {«~La Litt{é}rature Num{é}rique n'existe Pas.
{La} Litt{é}rarit{é} Au Prisme de l'imaginaire M{é}diatique
Contemporain:~»} \emph{Communication \& langages} N{\textdegree} 205
(3):5‑27. \url{https://doi.org/10.3917/comla1.205.0005}.

\leavevmode\vadjust pre{\hypertarget{ref-monjour_fait_2016}{}}%
Monjour, Servanne, Marcello Vitali Rosati, et Gérard Wormser. 2016.
{«~{Le fait litt{é}raire au temps du num{é}rique}~»}. \emph{Sens
public}, décembre.

\leavevmode\vadjust pre{\hypertarget{ref-montinari_nietzsche_1975}{}}%
Montinari, Mazzino. 1975. {«~Nietzsche {Briefwechsel}. {Kritische
Gesamtausgabe}~»}. \emph{Nietzsche-Studien} 3 (1).
\url{https://doi.org/10.1515/9783110244243.374}.

\leavevmode\vadjust pre{\hypertarget{ref-motion_cinder_2010}{}}%
Motion, Andrew. 2010a. \emph{The {Cinder Path}}. {London}: {Faber}.

\leavevmode\vadjust pre{\hypertarget{ref-motion_nox_2010}{}}%
Motion, Andrew. 2010b. {«~Nox by {Anne Carson}~»}. \emph{The Guardian},
juillet.

\leavevmode\vadjust pre{\hypertarget{ref-mouawad_incendies_2015}{}}%
Mouawad, Wajdi. 2015. \emph{{Incendies}}. {Montreal}: {Lem{é}ac}.

\leavevmode\vadjust pre{\hypertarget{ref-mouralis_les_2011}{}}%
Mouralis, Bernard, et Anthony Mangeon. 2011. \emph{Les
Contre-Litt{é}ratures}. "{Fictions} Pensantes". {Paris}: {Hermann}.

\leavevmode\vadjust pre{\hypertarget{ref-mucelli_fabbrica_2023}{}}%
Mucelli, Elena, et Francesco Gulinello. 2023. \emph{{La fabbrica
diffusa: produzione e architettura a Cesena}}. {Macerata}: {Quodlibet}.

\leavevmode\vadjust pre{\hypertarget{ref-muller_intermedialitat_1996}{}}%
Müller, Jürgen E. 1996. \emph{{Intermedialit{ä}t: Formen moderner
kultureller Kommunikation}}. {Film und Medien in der Diskussion} 8.
{M{ü}nster}: {Nodus-Publ}.

\leavevmode\vadjust pre{\hypertarget{ref-mumford_technics_1937}{}}%
Mumford, Lewis. 1937. {«~Technics and {Civilization}~»}. \emph{The
Journal of Nervous and Mental Disease} 86 (1):111‑12.
\url{https://doi.org/10.1097/00005053-193707000-00053}.

\leavevmode\vadjust pre{\hypertarget{ref-nail_being_2019}{}}%
Nail, Thomas. 2019. \emph{Being and {Motion}}. {New York, NY, United
States of America}: {Oxford University Press}.

\leavevmode\vadjust pre{\hypertarget{ref-nietzsche_voyageur_1880}{}}%
Nietzsche, Friedrich. 1880. \emph{{Le voyageur et son ombre}}.
{Culturea}.

\leavevmode\vadjust pre{\hypertarget{ref-nietzsche_correspondance_1986}{}}%
Nietzsche, Friedrich. 1986a. \emph{{Correspondance I : Juin 1850 - Avril
1869}}. Édité par Giorgio Colli et Mazzino Montinari. Traduit par
Henri-Alexis Baatsch, Jean Bréjoux, et Maurice de Gandillac. {Paris}:
{Gallimard}.

\leavevmode\vadjust pre{\hypertarget{ref-nietzsche_correspondance_1986-1}{}}%
Nietzsche, Friedrich. 1986b. \emph{{Correspondance II : Avril 1869 -
D{é}cembre 1874}}. Édité par Giorgio Colli et Mazzino Montinari. Traduit
par Jean Bréjoux et Maurice de Gandillac. {Paris}: {Gallimard}.

\leavevmode\vadjust pre{\hypertarget{ref-nietzsche_correspondance_2008}{}}%
Nietzsche, Friedrich. 2008. \emph{{Correspondance III : Janvier 1875 -
D{é}cembre 1879}}. Édité par Giorgio Colli et Mazzino Montinari. Traduit
par Jean Lacoste. {Paris}: {Gallimard}.

\leavevmode\vadjust pre{\hypertarget{ref-nowell_thomas_1960}{}}%
Nowell, Elizabeth. 1960. \emph{Thomas {Wolfe}: {A Biography}}. First
Edition. {Doubleday}.

\leavevmode\vadjust pre{\hypertarget{ref-cd09298df15b44cba37f59f00264ea23}{}}%
Nyhan, Julianne, et Melissa Terras. 2017. {«~Uncovering "{Hidden}"
{Contributions} to the {History} of {Digital Humanities}: {The Index
Thomisticus}' {Female Keypunch Operators}~»}. In \emph{Digital
{Humanities Conference} 2017}. {Montr{é}al}.

\leavevmode\vadjust pre{\hypertarget{ref-oresme_livre_1968}{}}%
Oresme, Nicole. 1968. \emph{Le {Livre} Du Ciel Et Du Monde}. Édité par
A. D. Menut et A. J. Denomy. {Madison, Wisconsin}: {The University of
Wisconsin Press}.

\leavevmode\vadjust pre{\hypertarget{ref-oulipo_litterature_1988}{}}%
Oulipo. 1988. \emph{{La Litt{é}rature potentielle : cr{é}ations,
re-cr{é}ations, r{é}cr{é}ations}}. {Collection Folio} 95. {Paris}:
{Gallimard}.

\leavevmode\vadjust pre{\hypertarget{ref-oulipo_atlas_2003}{}}%
Oulipo, éd. 2003. \emph{{Atlas de litt{é}rature potentielle}}.
{Collection folio Essais} 109. {Paris}: {Gallimard}.

\leavevmode\vadjust pre{\hypertarget{ref-oulipo_abrege_2005}{}}%
Oulipo, éd. 2005. \emph{{Abr{é}g{é} de litt{é}rature potentielle}}.
{Mille et une nuits} 379. {Paris}: {{É}ditions Mille et Une Nuits}.

\leavevmode\vadjust pre{\hypertarget{ref-palleau-papin_nox_2014}{}}%
Palleau-Papin, Françoise. 2014. {«~Nox: {Anne Carson}'s {Scrapbook
Elegy}~»}.

\leavevmode\vadjust pre{\hypertarget{ref-parikka_what_2012}{}}%
Parikka, Jussi. 2012. \emph{What {Is Media Archaeology}?} {Cambridge, UK
; Malden, MA}: {Polity Press}.

\leavevmode\vadjust pre{\hypertarget{ref-parinello_sinistra_2008}{}}%
Parinello, Giacomo. 2008. {«~{La sinistra rivoluzionaria italiana dopo
il Sessantotto Esperienze, orizzonti, linguaggi}~»}.
\emph{Storicamente}. \url{https://doi.org/10.1473/stor334}.

\leavevmode\vadjust pre{\hypertarget{ref-paveau_ce_2015}{}}%
Paveau, Marie-Anne. 2015a. {«~Ce Qui s'{é}crit Dans Les Univers
Num{é}riques : {Mati{è}res} Technolangagi{è}res et Formes
Technodiscursives~»}. \emph{Itin{é}raires}, nᵒ 2014-1 (janvier).
\url{https://doi.org/10.4000/itineraires.2313}.

\leavevmode\vadjust pre{\hypertarget{ref-paveau_presentation._2015}{}}%
Paveau, Marie-Anne. 2015b. {«~{Pr{é}sentation. Les textes num{é}riques
sont-ils des textes ?}~»} \emph{Itin{é}raires. Litt{é}rature, textes,
cultures}, nᵒ 2014-1 (février).

\leavevmode\vadjust pre{\hypertarget{ref-perec_infra-ordinaire_1989}{}}%
Perec, Georges. 1989. \emph{L'infra-Ordinaire}. La {Librairie} Du {XXe}
Si{è}cle. {Paris}: {{É}ditions du Seuil}.

\leavevmode\vadjust pre{\hypertarget{ref-pestre_introduction_2006}{}}%
Pestre, Dominique. 2006. \emph{{Introduction aux Science Studies}}. 1er
tirage. {Collection Rep{è}res sociologie} 449. {Paris}: {La
D{é}couverte}.

\leavevmode\vadjust pre{\hypertarget{ref-petit_lecriture_2017}{}}%
Petit, Victor, et Serge Bouchardon. 2017. {«~{L'{é}criture num{é}rique
ou l'{é}criture selon les machines. Enjeux philosophiques et
p{é}dagogiques}~»}. \emph{Communication \& langages} 191 (1). {Paris}:
{NecPlus}:129‑48. \url{https://doi.org/10.3917/comla.191.0129}.

\leavevmode\vadjust pre{\hypertarget{ref-phillips_app_2010}{}}%
Phillips, Tom. 2010. {«~The {App} of {A Humument}~»}.

\leavevmode\vadjust pre{\hypertarget{ref-pinson_imaginaire_2012}{}}%
Pinson, Guillaume. 2012. {«~{L'imaginaire m{é}diatique. R{é}flexions sur
les repr{é}sentations du journalisme au XIXe si{è}cle}~»}.
\emph{COnTEXTES. Revue de sociologie de la litt{é}rature}, nᵒ 11.
{Groupe de contact F.N.R.S. COnTEXTES}.
\url{https://doi.org/10.4000/contextes.5306}.

\leavevmode\vadjust pre{\hypertarget{ref-pohjoisen_kulttuuri-instituutti__institute_for_northern_culture_ingold_2013}{}}%
Pohjoisen kulttuuri-instituutti Institute for Northern Culture. 2013.
{«~Ingold -- {Thinking} through {Making}~»}.

\leavevmode\vadjust pre{\hypertarget{ref-ponge_methodes_1989}{}}%
Ponge, Francis. 1989. \emph{{M{é}thodes}}. {Collection folio Essais}
107. {Paris}: {Gallimard}.

\leavevmode\vadjust pre{\hypertarget{ref-ponge_fabrique_1990}{}}%
Ponge, Francis. 1990. \emph{{La fabrique du pr{é}}}. 2. Aufl. {Les
Sentiers de la cr{é}ation} 11. {Gen{è}ve}: {Skira}.

\leavevmode\vadjust pre{\hypertarget{ref-ponge_comment_1997}{}}%
Ponge, Francis. 1997. \emph{{Comment une figue de paroles et pourquoi}}.
Édité par Jean-Marie Gleize. {GF} 901. {Paris}: {Flammarion}.

\leavevmode\vadjust pre{\hypertarget{ref-ponge_oeuvres_1999}{}}%
Ponge, Francis. 1999a. \emph{{Œ}uvres Compl{è}tes}. Édité par Bernard
Beugnot. Biblioth{è}que de La {Pl{é}iade} 453-487. {Paris}: {Gallimard}.

\leavevmode\vadjust pre{\hypertarget{ref-ponge_oeuvres_1999-1}{}}%
Ponge, Francis. 1999b. \emph{{{Œ}uvres compl{è}tes}}. Édité par Michel
Collot. {Biblioth{è}que de la Pl{é}iade} 453. {Paris}: {Gallimard}.

\leavevmode\vadjust pre{\hypertarget{ref-ponge_pages_2005}{}}%
Ponge, Francis. 2005. \emph{Pages d'atelier, 1917-1982}. Édité par
Bernard Beugnot. Les Cahiers de La {NRF}. {Paris}: {Gallimard}.

\leavevmode\vadjust pre{\hypertarget{ref-ponge_entretiens_1967}{}}%
Ponge, Francis, et Philippe Sollers. 1967. \emph{{Entretiens de Francis
Ponge avec Philippe Sollers}}. 2e {é}d. {Points}. {Paris}: {{É}ditions
du Seuil}.

\leavevmode\vadjust pre{\hypertarget{ref-portanova_moving_2013}{}}%
Portanova, Stamatia. 2013. \emph{Moving {Without} a {Body}: {Digital
Philosophy} and {Choreographic Thought}}. Technologies of Lived
Abstraction. {Cambridge, Massachusetts}: {The MIT Press}.

\leavevmode\vadjust pre{\hypertarget{ref-puff_contrainte_2004}{}}%
Puff, Jean-François. 2004. {«~{La contrainte et la r{è}gle}~»}.
\emph{Po{é}tique} 140 (4). {Paris}: {Le Seuil}:455‑65.
\url{https://doi.org/10.3917/poeti.140.0455}.

\leavevmode\vadjust pre{\hypertarget{ref-queneau_cent_1961}{}}%
Queneau, Raymond. 1961. \emph{{Cent mille milliards de po{è}mes}}. Édité
par François Le Lionnais. {Paris}: {Gallimard}.

\leavevmode\vadjust pre{\hypertarget{ref-rainer_maria_rilke_primal_1919}{}}%
Rainer Maria Rilke. 1919. \emph{Primal {Sound}}. {Juan Andr{é}s M.
Fuentes}.

\leavevmode\vadjust pre{\hypertarget{ref-rajewsky_intermediality_2005}{}}%
Rajewsky, Irina. 2005. {«~Intermediality, {Intertextuality}, and
{Remediation}: {A Literary Perspective} on {Intermediality}~»}.
\emph{Interm{é}dialit{é}s : Histoire Et Th{é}orie Des Arts, Des Lettres
Et Des Techniques / Intermediality: History and Theory of the Arts,
Literature and Technologies}, nᵒ 6. {Centre de recherche sur
l'interm{é}dialit{é}}:43‑64. \url{https://doi.org/10.7202/1005505ar}.

\leavevmode\vadjust pre{\hypertarget{ref-raunig_modulation_2009}{}}%
Raunig, Gerald. 2009. {«~In {Modulation Mode}: {Factories} of
{Knowledge}~»}. Traduit par Aileen Derieg. \emph{Knowledge production
and its discontents}.

\leavevmode\vadjust pre{\hypertarget{ref-raunig_factories_2013}{}}%
Raunig, Gerald. 2013. \emph{Factories of {Knowledge}, {Industries} of
{Creativity}}. Édité par Aileen Derieg et Antonio Negri. Semiotext(e)
Intervention Series 15. {Los Angeles, CA. : Cambridge, MA}:
{Semiotext(e) ; distributed by the MIT Press}.

\leavevmode\vadjust pre{\hypertarget{ref-reggiani_rhetoriques_1999}{}}%
Reggiani, Christelle. 1999. {«~{Rh{é}toriques de la contrainte}~»}.
{Dix-neuf-vingt}. Thèse de doctorat, {Saint-Pierre-du-Mont}: {É}d.
Interuniversitaires.

\leavevmode\vadjust pre{\hypertarget{ref-reggiani_oulipo_2016}{}}%
Reggiani, Christelle, et Alain Schaffner, éd. 2016. \emph{Oulipo, Mode
d'emploi}. Litt{é}rature de Notre Si{è}cle 60. {Paris}: {Honor{é}
Champion {é}diteur}.

\leavevmode\vadjust pre{\hypertarget{ref-ribeiro_beyond_2020}{}}%
Ribeiro, Marco Tulio, Tongshuang Wu, Carlos Guestrin, et Sameer Singh.
2020. {«~Beyond {Accuracy}: {Behavioral Testing} of {NLP Models} with
{CheckList}~»}. In \emph{Proceedings of the 58th {Annual Meeting} of the
{Association} for {Computational Linguistics}}, 4902‑12. {Online}:
{Association for Computational Linguistics}.
\url{https://doi.org/10.18653/v1/2020.acl-main.442}.

\leavevmode\vadjust pre{\hypertarget{ref-ricoeur_temps_1983}{}}%
Ricœur, Paul. 1983. \emph{{Temps et r{é}cit. 1}}. {Paris}: {{É}ditions
du Seuil}.

\leavevmode\vadjust pre{\hypertarget{ref-robin_golem_1997}{}}%
Robin, Régine. 1997. \emph{Le {Golem} de l'ecriture : De l'autofiction
Au Cybersoi}. Th{é}orie et Litt{é}rature. {Montr{é}al, Qu{é}bec}: {XYZ
{é}diteur}.

\leavevmode\vadjust pre{\hypertarget{ref-rockwell_face_2013}{}}%
Rockwell, Geoffrey, Stan Ruecker, Jennifer Windsor, Mihaela Ilovan, et
Daniel Sondheim. 2013. {«~The {Face} of {Interface}: {Studying
Interface} to the {Scholarly Corpus} and {Edition}~»}. \emph{Scholarly
and Research Communication} 3 (4).
\url{https://doi.org/10.22230/src.2012v3n4a56}.

\leavevmode\vadjust pre{\hypertarget{ref-royer_16_2017}{}}%
Royer, Marine, et Marie-Julie Catoir-Brisson. 2017. {«~{Le design du
livre en contexte num{é}rique : conversation avec Ren{é}e Bourassa}~»}.
In \emph{{Design \& innovation dans la cha{î}ne du livre. {É}crire,
{é}diter, lire {à} l'{è}re num{é}rique}}, édité par Stéphane Vial,
177‑81.{{\guillemotleft}Hors collection {\guillemotright}}. {Paris}:
{Presses Universitaires de France}.

\leavevmode\vadjust pre{\hypertarget{ref-ruffel_litterature_2010}{}}%
Ruffel, David. 2010. {«~{Une litt{é}rature contextuelle}~»}.
\emph{Litterature} 160 (4). {Armand Colin}:61‑73.

\leavevmode\vadjust pre{\hypertarget{ref-ruffel_brouhaha_2016}{}}%
Ruffel, Lionel. 2016. \emph{Brouhaha : {Les} Mondes Du Contemporain}.
{Lagrasse}: {Verdier}.

\leavevmode\vadjust pre{\hypertarget{ref-saemmer_matieres_2007}{}}%
Saemmer, Alexandra. 2007. \emph{Mati{è}res Textuelles Sur Support
Num{é}rique}. Travaux / {Centre Interdisciplinaire} d'{{É}tude} et de
{Recherches} Sur l'{Expression Contemporaine} 132. {Saint-{É}tienne}:
{Publications de l'Universit{é} de Saint-{É}tienne}.

\leavevmode\vadjust pre{\hypertarget{ref-saemmer_hypertexte_2015}{}}%
Saemmer, Alexandra. 2015. {«~{Hypertexte et narrativit{é}}~»}.
\emph{Critique} 819--820 (8-9):637‑52.

\leavevmode\vadjust pre{\hypertarget{ref-samoyault_intertextualite:_2014}{}}%
Samoyault, Tiphaine, et Henri Mitterand. 2014.
\emph{{L'intertextualit{é} : M{é}moire de la litt{é}rature}}. {Paris}:
{A. Colin}.

\leavevmode\vadjust pre{\hypertarget{ref-san_martin_surface_2016}{}}%
San Martin, Caroline, et Karine Bouchy. 2016. {«~Surface, Copr{é}sence :
Circulations. {The Pillow Book} de {Peter Greenaway}~»}. \emph{Lignes de
fuite} 02.

\leavevmode\vadjust pre{\hypertarget{ref-saporta_composition_1962}{}}%
Saporta, Marc. 1962. \emph{Composition N{\textdegree} 1}. {{É}ditions du
Seuil}.

\leavevmode\vadjust pre{\hypertarget{ref-saporta_composition_2011}{}}%
Saporta, Marc. 2011. \emph{Composition No. 1: {This Book Can Be Read} in
{Any Order}}. Édité par Salvador Plascencia. {London}: {Visual
Editions}.

\leavevmode\vadjust pre{\hypertarget{ref-sauret_revue_2020}{}}%
Sauret, Nicolas. 2020. {«~{De la revue au collectif : la conversation
comme dispositif d'{é}ditorialisation des communaut{é}s savantes en
lettres et sciences humaines}~»}. \{Th\{\textbackslash`e\}se de
doctorat\}, {Montr{é}al, Canada}: Universit{é} de Montr{é}al.

\leavevmode\vadjust pre{\hypertarget{ref-schreibman_companion_2011}{}}%
Schreibman, Susan, éd. 2011. \emph{A {Companion} to {Digital
Humanities}}. Nachdr. Blackwell Companions to Literature et Culture 26.
{Malden, Mass.}: {Blackwell Publ}.

\leavevmode\vadjust pre{\hypertarget{ref-schroter_four_2012}{}}%
Schröter, Jens. 2012. {«~Four {Models} of {Intermediality}~»}.
\emph{Travels in Intermedia{[}lity{]}: Reblurring the Boundaries},
janvier, 15‑36.

\leavevmode\vadjust pre{\hypertarget{ref-schwabe_kontoristin_1902}{}}%
Schwabe, Jenny. 1902. \emph{Kontoristin: {Forderungen}, {Leistungen},
{Aussichten} in Diesem {Berufe}}. {Leipzig}.

\leavevmode\vadjust pre{\hypertarget{ref-sehgal_evoking_2011}{}}%
Sehgal, Parul. 2011. {«~Evoking the {Starry Lad Her Brother Was}~»}.
\emph{The Irish Times}, mars.

\leavevmode\vadjust pre{\hypertarget{ref-sene_fragments_nodate}{}}%
Séné, Joachim. s.~d. {«~Fragments, Chutes et Cons{é}quences.~»}
\emph{Fragments, chutes et cons{é}quences}. http://jsene.net/. Consulté
le 4 février 2020.

\leavevmode\vadjust pre{\hypertarget{ref-shakespeare_romeo_2008}{}}%
Shakespeare, William. 2008. \emph{Romeo and {Juliet}}. Édité par Brian
Gibbons et Richard Proudfoot. Nachdr. The {Arden Shakespeare} / General
Ed.: {Richard Proudfoot}. {London}: {Arden Shakespeare}.

\leavevmode\vadjust pre{\hypertarget{ref-shillingsburg_scholarly_1996}{}}%
Shillingsburg, Peter L. 1996. \emph{Scholarly Editing in the Computer
Age: Theory and Practice}. 3rd ed. Editorial Theory et Literary
Criticism. {Ann Arbor}: {University of Michigan Press}.

\leavevmode\vadjust pre{\hypertarget{ref-shonagon_notes_2009}{}}%
Shōnagon, Sei. 2009. \emph{{Notes de chevet}}. Édité par André Beaujard.
{Connaissance de l'Orient} 5. {Paris}: {Gallimard Unesco}.

\leavevmode\vadjust pre{\hypertarget{ref-sibony_creation_2003}{}}%
Sibony, Daniel. 2003. {«~{Cr{é}ation et entre-deux}~»}. \emph{Che vuoi
?} 19 (1). {Paris}: {L'Harmattan}:39‑56.
\url{https://doi.org/10.3917/chev.019.0039}.

\leavevmode\vadjust pre{\hypertarget{ref-sinclair_application_2000}{}}%
Sinclair, Stéfan. 2000. {«~{Une application d'HyperPo, un logiciel
d'analyse de texte informatis{é}e {à} La Disparition de Georges
Perec}~»}. Thèse de doctorat, {Kingston, Ontario}: Queen's University.

\leavevmode\vadjust pre{\hypertarget{ref-sklovskij_art_2008}{}}%
Šklovskij, Viktor Borisovič. 2008. \emph{{L'art comme proc{é}d{é}}}.
Édité par Régis Gayraud. {Paris}: {{É}ditions Allia}.

\leavevmode\vadjust pre{\hypertarget{ref-smith_exceptional_2018}{}}%
Smith, Dominic. 2018. \emph{Exceptional {Technologies}: {A Continental
Philosophy} of {Technology}}. {London, UK}: {Bloomsbury Academic}.

\leavevmode\vadjust pre{\hypertarget{ref-sollers_entretiens_1970}{}}%
Sollers, Philippe. 1970. \emph{{Entretiens de Francis Ponge}}.
{Gallimard}.

\leavevmode\vadjust pre{\hypertarget{ref-souchier_image_1998}{}}%
Souchier, Emmanuël. 1998. {«~{L'image du texte pour une th{é}orie de
l'{é}nonciation {é}ditoriale}~»}. \emph{Les cahiers de mediologie} 6
(2):137‑45.

\leavevmode\vadjust pre{\hypertarget{ref-souchier__2012}{}}%
Souchier, Emmanuël. 2012. {«~La {\guillemotleft} Lettrure
{\guillemotright} {à} l'{é}cran {Lire} \& {É}crire Au Regard Des
M{é}dias Informatis{é}s~»}. \emph{Communication \& Langages} 2012
(174):85‑108. \url{https://doi.org/10.4074/S033615001201407x}.

\leavevmode\vadjust pre{\hypertarget{ref-souchier_carnaval_2015}{}}%
Souchier, Emmanuël. 2015. {«~{Le carnaval typographique de Balzac.
Premiers {é}l{é}ments pour une th{é}orie de l'irr{é}ductibilit{é}
s{é}miotique}~»}. \emph{Communication langages} 185 (3):3‑22.

\leavevmode\vadjust pre{\hypertarget{ref-stang_nox_2012}{}}%
Stang, Charles M. 2012. {«~"{Nox}" or the {Muteness} of {Things}~»}.
\emph{Havard Divinity Bulletin} 40 (1-2).

\leavevmode\vadjust pre{\hypertarget{ref-stein_geography_1993}{}}%
Stein, Gertrude. 1993. \emph{Geography and {Plays}}. {Madison}:
{University of Wisconsin Press}.

\leavevmode\vadjust pre{\hypertarget{ref-stumpel_vom_1985}{}}%
Stümpel, Rolf. 1985. \emph{{Vom Sekret{ä}r zur Sekret{ä}rin. Eine eine
Ausstellung zur Geschichte der Schreibmaschine und ihrer Bedeutung f{ü}r
den Beruf der Frau im B{ü}ro}}. {Mainz}: print; {Schmidt \& B{ö}dige}.

\leavevmode\vadjust pre{\hypertarget{ref-suchman_human-machine_2007}{}}%
Suchman, Lucille Alice. 2007. \emph{Human-{Machine Reconfigurations}:
{Plans} and {Situated Actions}}. 2nd ed. {Cambridge ; New York}:
{Cambridge University Press}.

\leavevmode\vadjust pre{\hypertarget{ref-swidzinski_art_1997}{}}%
Swidzinski, Jan. 1997. {«~{L'art comme art contextuel (manifeste)}~»}.
\emph{Inter} 68:46‑50.

\leavevmode\vadjust pre{\hypertarget{ref-sze_consolatory_2019}{}}%
Sze, Gillian. 2019. {«~The {Consolatory Fold}: {Anne Carson}'s {Nox} and
the {Melancholic Archive}~»}. \emph{Studies in Canadian Literature /
{É}tudes En Litt{é}rature Canadienne} 44 (1):66‑80.
\url{https://doi.org/10.7202/1066499ar}.

\leavevmode\vadjust pre{\hypertarget{ref-tanasescu_graphpoem_2022}{}}%
Tanasescu, Chris. 2022. {«~\#{GraphPoem} @ {DHSI}: {A Poetics} of
{Network Walks}, {Stigmergy}, and {Accident} in {Performance}~»}.
\emph{IDEAH} 3 (1). \url{https://doi.org/10.21428/f1f23564.e6beae69}.

\leavevmode\vadjust pre{\hypertarget{ref-tari_autonomie_2011}{}}%
Tarì, Marcello, et Étienne Dobenesque. 2011. \emph{{Autonomie ! Italie,
les ann{é}es 1970}}. {Paris}: {{É}ditions La Fabrique}.

\leavevmode\vadjust pre{\hypertarget{ref-tassi_predominance_2021}{}}%
Tassi, Philippe. 2021. {«~{La pr{é}dominance de la vue~: L'imprimerie,
la presse, l'affichage}~»}. In \emph{{Les m{é}dias et leurs fonctions}},
67‑96. {Questions de soci{é}t{é}}. {Caen}: {EMS Editions}.

\leavevmode\vadjust pre{\hypertarget{ref-ted_nelson_ted_2016}{}}%
Ted Nelson. 2016. {«~Ted {Nelson} in {Herzog}'s "{Lo} and {Behold}"~»}.

\leavevmode\vadjust pre{\hypertarget{ref-theval_poetique_2018}{}}%
Théval, Gaëlle. 2018. {«~{Une po{é}tique en retravail}~»}. \emph{Acta
Fabula}, nᵒ vol. 19, n{\textdegree} 5.

\leavevmode\vadjust pre{\hypertarget{ref-thomas_machinations_1986}{}}%
Thomas, Jean-Jacques. 1986. {«~Machinations Formelles : Sur
l'{Oulipo}~»}. \emph{L'Esprit Cr{é}ateur} 26 (4). {The Johns Hopkins
University Press}:71‑86. \url{https://www.jstor.org/stable/26284662}.

\leavevmode\vadjust pre{\hypertarget{ref-tourte__2018}{}}%
Tourte, Élise. 2018.{«~{{\guillemotleft} Contre la colle les uns les
autres {\guillemotright} : l'{é}preuve du contact chez Henri
Michaux}~»}. \emph{Les Lettres Romanes} 72 (3-4):385‑94.
\url{https://doi.org/10.1484/J.LLR.5.117223}.

\leavevmode\vadjust pre{\hypertarget{ref-turcotte_autobiographie_2017}{}}%
Turcotte, Élise. 2017. \emph{{Autobiographie de l'esprit : {é}crits
sauvages et domestiques}}. 2e {é}dition, revue et augment{é}e.
{L'Ouvroir}. {Montr{é}al}: {La M{è}che}.

\leavevmode\vadjust pre{\hypertarget{ref-turing_computing_1950}{}}%
Turing, A. M. 1950. {«~Computing {Machinery} and {Intelligence}~»}.
\emph{Mind} LIX (236):433‑60.
\url{https://doi.org/10.1093/mind/LIX.236.433}.

\leavevmode\vadjust pre{\hypertarget{ref-turner_typology_1977}{}}%
Turner, Eric G. 1977. \emph{The Typology of the Early Codex}. Haney
{Foundation Series} 18. {Philadelphia, Pa}: {Univ. of Pennsylvania Pr}.

\leavevmode\vadjust pre{\hypertarget{ref-turner_terms_1978}{}}%
Turner, Eric Gardner. 1978. \emph{The {Terms Recto} and {Verso}. {The
Anatomy} of the {Papyrus Roll}}. {Bruxelles : Fondation {é}gyptologique
reine {É}lisabeth}.

\leavevmode\vadjust pre{\hypertarget{ref-unesco_id_2019}{}}%
Unesco. 2019. {«~I'd {Blush If I Could}: {Closing Gender Divides} in
{Digital Skills Through Education}~»}.
https://unesdoc.unesco.org/ark:/48223/pf0000367416.page=1.

\leavevmode\vadjust pre{\hypertarget{ref-valery_variete_1930}{}}%
Valéry, Paul. 1930. \emph{{Vari{é}t{é} II}}. {Paris}: {Gallimard}.

\leavevmode\vadjust pre{\hypertarget{ref-valery_degas_1998}{}}%
Valéry, Paul. 1998. \emph{{Degas, danse, dessin}}. {Collection folio
Essais} 323. {Paris}: {Gallimard}.

\leavevmode\vadjust pre{\hypertarget{ref-vandendorpe_du_1999}{}}%
Vandendorpe, Christian. 1999. \emph{{Du papyrus {à} l'hypertexte : Essai
sur les mutations du texte et de la lecture}}. {Sciences et
soci{é}t{é}}. {Paris}: {La D{é}couverte}.

\leavevmode\vadjust pre{\hypertarget{ref-vian_robot_1972}{}}%
Vian, Boris. 1972. {«~{Un robot ne nous fait pas peur}~»}. In \emph{{Je
voudrais pas crever}}. {Hachette Le Livre de Poche}.

\leavevmode\vadjust pre{\hypertarget{ref-villiers_de_lisle-adam_eve_1993}{}}%
Villiers de L'Isle-Adam, Auguste de. 1993. \emph{{L' {è}ve future}}.
Édité par Alan W. Raitt. {Collection Folio} 2498. {Paris}: {Gallimard}.

\leavevmode\vadjust pre{\hypertarget{ref-vitali_rosati_editorialization:_2018}{}}%
Vitali Rosati, Marcello. 2018. \emph{On {Editorialization}: {Structuring
Space} and {Authority} in the {Digital Age}}.

\leavevmode\vadjust pre{\hypertarget{ref-vitali-rosati_what_2016}{}}%
Vitali-Rosati, Marcello. 2016. {«~What {Is Editorialization}?~»}
\emph{Sens Public}, janvier.

\leavevmode\vadjust pre{\hypertarget{ref-vitali-rosati_mais_2018}{}}%
Vitali-Rosati, Marcello. 2018. {«~Mais O{ù} Est Pass{é} Le R{é}el ?
{Profils}, Repr{é}sentations Et M{é}taontologie~»}. \emph{Muse Medusa}
6.

\leavevmode\vadjust pre{\hypertarget{ref-vitali-rosati_quest-ce_2020}{}}%
Vitali-Rosati, Marcello. 2020. {«~{Qu'est-ce que l'{é}criture
num{é}rique ?}~»} \emph{Corela. Cognition, repr{é}sentation, langage},
nᵒ HS-33 (novembre). {Universit{é} de Poitiers}.
\url{https://doi.org/10.4000/corela.11759}.

\leavevmode\vadjust pre{\hypertarget{ref-vitali-rosati_fait_2021}{}}%
Vitali-Rosati, Marcello. 2021. {«~{Le fait num{é}rique comme
"conjonctures m{é}diatrices"}~»}. \emph{Communication \& langages}
208--209 (2-3). {Paris cedex 14}: {Presses Universitaires de
France}:155‑70. \url{https://doi.org/10.3917/comla1.208.0155}.

\leavevmode\vadjust pre{\hypertarget{ref-vitali-rosati_eloge_2024}{}}%
Vitali-Rosati, Marcello. 2024. \emph{{É}loge Du Bug}. Zone. {La
D{é}couverte}.

\leavevmode\vadjust pre{\hypertarget{ref-bourassa_editorialiser_2021}{}}%
Vitali-Rosati, Marcello, et Margot Mellet. 2021. {«~{{É}ditorialiser
l'Anthologie grecque. L'API comme livre num{é}rique}~»}. In \emph{{Le
livre en contexte num{é}rique: Un d{é}fi de design}}. {ARCANES,
Qu{é}bec}.

\leavevmode\vadjust pre{\hypertarget{ref-vitali-rosati_lepopee_2021}{}}%
Vitali-Rosati, Marcello, Margot Mellet, Servanne Monjour, Antoine
Fauchié, Timothée Guicherd, David Larlet, et Enrico Agostini-Marchese.
2021. {«~{L'{é}pop{é}e num{é}rique de l'Anthologie grecque : entre
questions {é}pist{é}mologiques, mod{è}les techniques et dynamiques
collaboratives}~»}. \emph{Sens public}, nᵒ SP1596 (juillet).
{D{é}partement des litt{é}ratures de langue fran{ç}aise}.

\leavevmode\vadjust pre{\hypertarget{ref-vitali-rosati_editorializing_2019}{}}%
Vitali-Rosati, Marcello, Servanne Monjour, Joana Casenave, Elsa
Bouchard, et Margot Mellet. 2019. {«~Editorializing the {Greek
Anthology}: {The Palatin Manuscript} as a {Collective Imaginary}~»}.
\emph{Digital Humanities Quarterly} 014 (1).

\leavevmode\vadjust pre{\hypertarget{ref-vitruve_architecture_1986}{}}%
Vitruve, Marcus. 1986. \emph{{De l'Architecture. Livre X}}. Les Belles
Lettres. {Gallimard}.

\leavevmode\vadjust pre{\hypertarget{ref-wajcman_technofeminism_2004}{}}%
Wajcman, Judy. 2004. \emph{{TechnoFeminism}}. {Cambridge ; Malden, MA}:
{Polity}.

\leavevmode\vadjust pre{\hypertarget{ref-wardrip-fruin_newmediareader_2003}{}}%
Wardrip-Fruin, Noah, Nick Montfort, et Michael Crumpton. 2003. \emph{The
{NewMediaReader}}. {Cambridge, Mass}: {MIT Press}.

\leavevmode\vadjust pre{\hypertarget{ref-web_deleuze_gilles_2019}{}}%
Web Deleuze. 2019. {«~Gilles {Deleuze} - {Sur Spinoza} - {S{é}ance} 1 -
{Cours} Du 2 D{é}cembre 1980~»}.

\leavevmode\vadjust pre{\hypertarget{ref-wershler_xenotext_2012}{}}%
Wershler, Darren. 2012. {«~The {Xenotext Experiment}, {So Far}~»}.
\emph{Canadian Journal of Communication} 37 (1). {University of Toronto
Press}:43‑60. \url{https://doi.org/10.22230/cjc.2012v37n1a2526}.

\leavevmode\vadjust pre{\hypertarget{ref-wertheim_noulipian_2007}{}}%
Wertheim, Christine. 2007. \emph{The {Noulipian Analects}}. Édité par
Matias Viegerner. {Los Angeles}: {Les Figues Press}.

\leavevmode\vadjust pre{\hypertarget{ref-wittgenstein_philosophical_2000}{}}%
Wittgenstein, Ludwig. 2000. \emph{Philosophical {Investigations}: {The
English Text} of the {Third Edition}}. Édité par G. E. M. Anscombe. 3.
ed. {Englewood Cliffs, N.J}: {Prentice Hall}.

\leavevmode\vadjust pre{\hypertarget{ref-wolff_reading_2007}{}}%
Wolff, Mark. 2007. {«~Reading {Potential}: {The Oulipo} and the
{Meaning} of {Algorithms}~»}. \emph{Digital Humanities Quarterly} 001
(1).

\leavevmode\vadjust pre{\hypertarget{ref-wolterstorff_works_1980}{}}%
Wolterstorff, Nicholas. 1980. \emph{Works and {Worlds} of {Art}}.
Clarendon {Library} of {Logic} et {Philosophy}. {Oxford, New York}:
{Oxford University Press}.

\leavevmode\vadjust pre{\hypertarget{ref-wurth_re-vision_2013}{}}%
Wurth, Kiene Brillenburg. 2013. {«~Re-{Vision} as {Remediation}:
{Hypermediacy} and {Translation} in {Anne Carson}'s {Nox}~»}.
\emph{Image {[}and{]} Narrative} 14 (4):20‑33.

\leavevmode\vadjust pre{\hypertarget{ref-zacklad_transactions_2005}{}}%
Zacklad, Manuel. 2005. {«~{Transactions communicationnelles symboliques
: innovation et cr{é}ation de valeur dans les communaut{é}s
d'action}~»}. \emph{A para{î}tre in, Lorino, P., Teulier, R. (2005),
{\guillemotleft} Entre la connaissance et l'organisation, l'activit{é}
collective {\guillemotright}, Masp{é}ro, Paris.}, janvier.

\leavevmode\vadjust pre{\hypertarget{ref-zacklad_reseaux_2007}{}}%
Zacklad, Manuel. 2007. \emph{{R{é}seaux et communaut{é}s d'imaginaire
docum{é}diatis{é}es}}. {Peter Lang}.

\leavevmode\vadjust pre{\hypertarget{ref-gagnon-arguin_genre_2015}{}}%
Zacklad, Manuel. 2015. {«~Genre de Dispositifs de M{é}diation
Num{é}rique et R{é}gimes de Documentalit{é}~»}. In \emph{Les Genres de
Documents Dans Les Organisations, {Analyse} Th{é}orique et Pratique},
145‑83. {Qu{é}bex}: {PUQ}.

\leavevmode\vadjust pre{\hypertarget{ref-zali_aventure_1997}{}}%
Zali, Anne, Annie Berthier, et Bibliothèque nationale de France, éd.
1997. \emph{L'aventure Des {É}critures. {Naissances}}. {Paris}:
{Biblioth{è}que nationale de France}.

\leavevmode\vadjust pre{\hypertarget{ref-zielinski_deep_2006}{}}%
Zielinski, Siegfried. 2006. \emph{Deep {Time} of the {Media}: {Toward}
an {Archaeology} of {Hearing} and {Seeing} by {Technical Means}}.
Electronic Culture--History, Theory, Practice. {Cambridge, Mass}: {MIT
Press}.

\end{CSLReferences}
