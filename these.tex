%%
%% This is file `gabaritmem.tex',
%% generated with the docstrip utility.
%%
%% The original source files were:
%%
%% dms.dtx  (with options: `memoire,gabarit')
%% Example TeX file for the documentation
%% of the jurabib package
%% Copyright (C) 1999, 2000, 2001 Jens Berger
%% See dms.ins  for the copyright details.
%% 
%%% ====================================================================
%%%  @LaTeX-file{
%%%     filename        = "dms.dtx",
%%%     author    = "Nicolas Beauchemin, Damien Rioux-Lavoie, Victor Fardel, Jonathan Godin",
%%%     copyright = "Copyright (C) 2000 , DMS
%%%                  all rights reserved.  Copying of this file is
%%%                  authorized only if either:
%%%                  (1) you make absolutely no changes to your copy,
%%%                  including name; OR
%%%                  (2) if you do make changes, you first rename it
%%%                  to some other name.",
%%%     address   = "Département de Mathématiques et de Statistique",
%%%     telephone = "514-343-6705",
%%%     FAX       = "514-343-5700",
%%%     email     = "aide@dms.umontreal.ca (Internet)",
%%%     keywords  = "latex, amslatex, ams-latex, theorem",
%%%     abstract  = " Ce fichier est un package conçu pour être
%%%                  utilisé avec la version de LaTeX2e 1995/06/01. Il
%%%                  est prévue pour la classe ``amsbook''. Il en
%%%                  modifie le format des pages, l'entête des
%%%                  sections, etc, afin d'être  conforme au modèle de
%%%                  mémoire de maîtrise de l'Université de
%%%                  Montréal. Finalement ce fichier est grandement
%%%                  inspiré du fichier amsclass.dtx.",
%%%     docstring = "The checksum field contains: CRC-16 checksum,
%%%                  word count, line count, and character count, as
%%%                  produced by Robert Solovay's checksum utility."}
%%%  ====================================================================

%% Pour voir les accents de ce fichier, assurez-vous que votre
%% éditeur de texte lise le fichier en utf-8!

%% La classe <dms> est construite au-dessus de <amsbook>, donc
%% <amsmath>, <amsfonts> et <amsthm> sont automatiquement chargés.
%% Pour un mémoire
% \documentclass[12pt,twoside,maitrise]{dms}
%% Pour une thèse
\documentclass[12pt,twoside,phd]{dms}

\usepackage[utf8]{inputenc} %Obligatoires
\usepackage[T1]{fontenc}    %

%% <lmodern> incorpore les fontes en T1, pour
%% faciliter le dépôt final. Ceci n'est pas la
%% seule option :
%%  1. Si cm-super est installé, vous pouvez enlever <lmodern>
%%     (à ce moment, la police est un peu plus fidèle
%%      au Computer Modern orginal);
%%  2. Si vous avez une police préférée, par exemple,
%%     <times> ou <euler> ou <mathpazo> (et bien d'autres),
%%     alors vous pouvez remplacer <lmodern> ci-bas.
%% Par contre, si vous faîtes face à un problème d'encapsulation
%% lors dépôt final, il se peut que la solution soit d'utiliser <lmodern>.
%% (Parfois le problème est au niveau de l'installation, donc
%%  essayez de compiler sur un autre ordinateur sur lequel vous êtes
%%  certain·e que l'installation est bonne.)
\usepackage{lmodern}
\DeclareSymbolFont{largesymbols}{OMX}{cmex}{m}{n}

%% Il n'est pas nécessaire d'utiliser <babel>, car
%% les commandes intégrées par la classe <dms>
%% \francais et \anglais font le travail. Néanmoins,
%% certains autres packages nécessitent <babel> (comme
%% <natbib>), donc simplement enlever les % devant <babel>
%% dans ce cas. Attention! Certains packages sont sensibles
%% à l'ordre dans lequel ils sont chargés.
% \francais % or
%%\anglais
%%
\usepackage[english,frenchb]{babel}

 % ENGLISH OPTION
 % If you call \anglais here before the \begin{document},
 % all the chater's header will be in english, even if you
 % call \francais. To change this, use
 % \entetedynamique

%% La commande \sloppy peut avoir des effets étranges sur les
%% lignes de certains paragraphes.  Dans ce cas, essayez \fussy
%% qui suppresse les effets de \sloppy.
%% (\fussy est normalement le comportement par défaut.)
%% On redéfinit \sloppy, pour tenter de réduire les comportements
%% étranges. Le seul changement apporté à la version originale
%% est la valeur de \tolerance.
\def\sloppy{%
  \tolerance 500%  %9999 dans LaTeX ordinaire, mauvaise idée.
  \emergencystretch 3em%
  \hfuzz .5pt
  \vfuzz\hfuzz}
\sloppy   %appel de \sloppy pour le document
%%\fussy  %ou \fussy

%% Packages utiles.
\usepackage{graphicx,amssymb,subfigure,icomma}
%% icomma       permet d'écrire les nombres décimaux en
%%                  français (p.ex. 1,23 plutôt que 1.23)
%% subfigure    simplifie l'inclusion de figures côtes-à-côtes

%% Packages parfois utiles.
%%\usepackage{dsfont,mathrsfs,color,url,verbatim,booktabs}
%% dsfont       symboles mathématiques \mathds
%% mathrsfs     plus de symboles mathématiques \mathscr
%% color        pour utiliser des couleurs (comparer avec <xcolor>)
%% url          permet l'écriture d'url
%% verbatim     pour écrire du code ou du texte tel quel
%% booktabs     plus de macros pour faire les tableaux
%%                  (voir documentation du package)

%% pour que la largeur de la légende des figures soit = \textwidth
\usepackage[labelfont=bf, width=\linewidth]{caption}

%% les 3 lignes suivante servent à l'affichage de l'index
%% dans le visionneur de pdf. <hyperref> et <bookmark>
%% devraient être les dernier package a être chargé,
%% donc chargez vos packages avant.
\usepackage{hyperref}  % Ajoute les hyperlien
\hypersetup{colorlinks=true,allcolors=black}
\usepackage{hypcap}   % Corrige la position du lien pour les images
\usepackage{bookmark} % Remédie à des petits problème
                      % de <hyperref> (important qu'il
                      % apparaisse APRÈS <hyperref>)

  % Enlever les commentaires du prochaine \hypersetup et
  % le remplir avec l'information pertinente.
  % Ceci ajoute des « méta-données » au pdf.  C'est optionnel,
  % mais recommandé. Vous pouvez voir ces méta-données en
  % ouvrant un visionneur de pdf et en cherchant les propriétés
  % du pdf. (Vous pouvez aussi tapez ' pdfinfo <nom-du-pdf> '
  % dans un terminal.) Ces données sont utiles, par exemple,
  % pour augmenter les chances qu'un algorithme de recherche
  % trouve votre document sur Internet, une fois diffusé.
%%\hypersetup{
%%  pdftitle = {Titre de la thèse / du mémoire},
%%  pdfauthor = {auteur·e},
%%  pdfsubject = {Ex: Transformation de Fourier ; régressions linéaires ; ... },
%%  pdfkeywords = {Ex: mathématiques, statistiques, groupes, variables aléatoires,...}
%%}

%% Définition des environnements utiles pour un mémoire scientifique.
%% La numérotation est laissée à la discrétion de l'auteur·e. L'exemple
%% illustré ici produit « Définition x.y.z »
%%   x = no. chapitre
%%   y = no. section
%%   z = no. définition
%% et la numérotation des corollaires, définitions, etc. se fait
%% successivement.
%%
%% Les macros \<type>name sont telles qu'ils suivent
%% la langue actuelle. (P.ex. si \francais est utilisé,
%% alors \begin{theo} va faire un Théorème et si \anglais
%% est utilisé, \begin{theo} fera un Theorem.)
%%
\newtheorem{cor}{\corollaryname}[section]
\newtheorem{deff}[cor]{\definitionname}
\newtheorem{ex}[cor]{\examplename}
\newtheorem{lem}[cor]{\lemmaname}
\newtheorem{prop}[cor]{Proposition}
\newtheorem{rem}[cor]{\remarkname}
\newtheorem{theo}[cor]{\theoremname}
\theoremstyle{definition}
\newtheorem{algo}[cor]{\algoname}
%% NOTE : Il peut être commode de redéfinir \the<type> pour
%% obtenir la numérotation désirée. Par exemple, pour
%% que les corollaires soit numérotés #section.#sous-section.#sous-sous-section.#paragraphe.#corollaire,
%% on fait
%% \renewcommand\thecor{\theparagraph.\arabic{cor}}

%%%
%%% Si vous préférez que les corollaires, définitions, théorèmes,
%%% etc. soient numérotés séparément, utilisez plutôt un bloc de
%%% commandes de la forme :
%%%

%%\newtheorem{cor}{\corollaryname}[section]
%%\newtheorem{deff}{\definitionname}[section]
%%\newtheorem{ex}{\examplename}[section]
%%\newtheorem{lem}{\lemmaname}[section]
%%\newtheorem{prop}{Proposition}[section]
%%\newtheorem{rem}{\remarkname}[section]
%%\newtheorem{theo}{\theoremname}[section]

%%
%% Numérotation des équations par section
%% et des  tableaux et figures par chapitre.
%% Ceci peut être modifié selon les préférences de l'utilisateur.
\numberwithin{equation}{section}
\numberwithin{table}{chapter}
\numberwithin{figure}{chapter}

%%
%% Si on veut faire un index, il faut décommenter la ligne
%% suivante. Ajouter des mots à l'index avec la commande \index{mot cle} au
%% fur et à mesure dans le texte.  Compiler, puis taper la commande
%% makeindex pour creer les indexs.  Après une nouvelle compilation,
%% vous aurez votre index.
%%

%%\makeindex

%% Il est obligatoire d'écrire à double interligne
%% ou à interligne et demi. On peut soit utiliser
%% le package <setspace> ou \baselinestretch.
%% Le package a tendance a créé des grands espaces blancs,
%% le gabarit décourage son utilisation, mais il en
%% reste à la discrétion de l'utilisateur·e.
%% \usepackage[onehalfspacing]{setspace}
 % ou
\renewcommand{\baselinestretch}{1.286} %Interligne et demi (environ 18pt (12pt+6pt) entre les lignes)

%%%%%%%%%%%%%%%%%%%%%%%%%%%%%%%%%%%%%%%%%%%%%%%%%%%%%%%%%%%%
%%%%%%%%%%%%%%%%%%%%%%%%%%%%%%%%%%%%%%%%%%%%%%%%%%%%%%%%%%%%
%%%%%%%%%%                                     %%%%%%%%%%%%%
%%%%%%%%%% D é b u t    d u    d o c u m e n t %%%%%%%%%%%%%
%%%%%%%%%%                                     %%%%%%%%%%%%%
%%%%%%%%%%%%%%%%%%%%%%%%%%%%%%%%%%%%%%%%%%%%%%%%%%%%%%%%%%%%
%%%%%%%%%%%%%%%%%%%%%%%%%%%%%%%%%%%%%%%%%%%%%%%%%%%%%%%%%%%%
\begin{document}

%%
%% Voici des options pour annoter les différentes versions de votre
%% mémoire. La commande \brouillon imprime, au bas de chacune des pages, la
%% date ainsi que l'heure de la dernière compilation de votre fichier.
%%
%%\brouillon
%%
%%
%% \version est la version de votre manuscrit
%%
\version{1}
\pagenumbering{roman}

%%------------------------------------------------- %
%%              pages i et ii                       %
%%------------------------------------------------- %

%%%
%%% Voici les variables à définir pour les deux premières pages de votre
%%% mémoire.
%%%

\title{Titre du mémoire ou de la thèse}

\author{Margot Mellet}

\copyrightyear{}

\department{Département de mathématiques et de statistique}

\date{\today} %Date du DÉPÔT INITIAL (ou du 2e dépôt s'il y a corrections majeures)

\sujet{Discipline}
%%\orientation{orientation}%Ce champ est optionnel
%%
%% Voici les disciplines possibles (voir avec votre directeur):
%% \sujet{statistique},
%% \sujet{mathématiques}, \orientation{mathématiques appliquées},
%% \orientation{mathématiques fondamentales}
%% \orientation{mathématiques de l'ingénieur} et
%% \orientation{mathématiques appliquées}

\president{Nom du président du jury}

\directeur{Nom du directeur de recherche}

%%\codirecteur{Nom du 1er codirecteur}         % s'il y a lieu
%%\codirecteurs{Nom du 2e codirecteur}         % s'il y a lieu

\membrejury{Nom du membre de jury}

%%\examinateur{Nom de l'examinateur externe}   %obligatoire pour la these

%% \membresjury{Deuxième membre du jury}  % s'il y a lieu

%%  \plusmembresjury{Troisième membre du jury}    % s'il y a lieu

 % Cette option existe encore, mais elle n'a plus sa place
 % dans la page titre. L'utiliser seulement si le directeur
 % insiste...
%%\repdoyen{Nom du représentant du doyen} %(thèse seulement)

%%
%% Fin des variables à définir. La commande \maketitle créera votre
%% page titre.

%% Pour mettre bouton qui mène à la page titre
%% dans le visionneur de pdf. Peut être enlever.
\pdfbookmark[chapter]{Couverture}{PageUn}

\maketitle

 % Pour générer la deuxième page titre, il faut appeler à nouveau \maketitle
 % Cette page est obligatoire.
\maketitle

%%------------------------------------------------- %
%%              pages iii                           %
%%------------------------------------------------- %

\francais

\chapter*{Résumé}



...sommaire et mots clés en français...

%%------------------------------------------------- %
%%              pages iv                            %
%%------------------------------------------------- %

\anglais
\chapter*{Abstract}

...summary and keywords in english...

%%------------------------------------------------- %
%%        page v --- Table de matieres              %
%%------------------------------------------------- %

 % Pour un mémoire en anglais, changer pour
 % \anglais. Noter qu'il faut une permission
 % pour écrire son mémoire en anglais.
%%\anglais
% \francais
 % \cleardoublepage termine la page actuel et force TeX
 % a poussé les éléments flottant (fig., tables, etc.) sur
 % la page (normalement TeX les garde en suspend jusqu'à ce
 % qu'il trouve un endroit approprié). Avec l'option <twoside>,
 % la commande s'assure que la prochaine page de texte est sur
 % le recto, pour l'impression. On l'utilise ici
 % pour que TeX sache que la table des matières etc. soit
 % sur la page qui suit.
%% TABLE DES MATIÈRES
\cleardoublepage
\pdfbookmark[chapter]{\contentsname}{toc}  % Crée un bouton sur
                                           % la bar de navigation
\tableofcontents
 % LISTE DES TABLES
\cleardoublepage
\phantomsection  % Crée une section invisible (utile pour les hyperliens)
\listoftables
 % LISTE DES FIGURES
\cleardoublepage
\phantomsection
\listoffigures

%%%%%%%%%%%%%%%%%%%%%%%%%%%%%%%%%%%%%
%% LISTE DES SIGLES ET ABRÉVIATION %
%%%%%%%%%%%%%%%%%%%%%%%%%%%%%%%%%%%%%
%% Il est obligatoire, selon les directives de la FESP,
%% pour une thèse ou un mémoire d'avoir une liste des sigles et
%% des abréviations.  Si vous considérez que de telles listes ne seraient pas
%% pertinentes (si, par exemple, vous n'utilisez aucun sigle ou abré.), son
%% inclusion ou omission est laissé à votre discrétion.  En cas de doute,
%% parlez-en à votre directeur de recherche, le coadministrateur ou au/à la
%% bibliothécaire.
%%
%% Le gabarit inclut un exemple d'une liste « fait à la main ».  Il existe des outils
%% plus sophistiqués si vous devez inclure une multitude de sigles et abréviations.
%% Par exemple, le package <glossaries> peut faire des index élaborés.  Comme
%% son utilisation est technique, il n'y a pas d'exemple directement dans ce gabarit.
%% On invite les gens qui aurait à l'utiliser à lire la documentation officielle,
%% soit en allant sur https://www.ctan.org/, soit en tapant dans un terminal :
%%
%% texdoc glossaries
%%

\chapter*{Liste des sigles et des abréviations}
 % Option de colonnes: definir \colun ou \coldeux
%%% Exemple
%%% \def\colun{\bf} % Première colonne en gras
%%% Pour numéroté les entrées, on peut faire
%%% \newcount\abbrlist
%%% \abbrlist=0
%%% \def\plusun{\global\advance\abbrlist by 1\relax}
%%% \def\colun{\plusun\the\abbrlist. }
%%\def\coldeux{\relax}
\begin{twocolumnlist}{.2\textwidth}{.7\textwidth}
  KQ-Methode & Méthode des moindres carrés, de l'allemand
      \textit{Methode der kleinsten Quadrate}\\
  MCMC  & Monte Carlo par chaînes de Markov, de l'anglais
      \textit{Markov Chain Monte Carlo} \\
  MSE  &  Erreur quadratique moyenne, de l'anglais \textit{Mean Square Error}\\
  NDR & Retract d'un voisinage, de l'anglais \textit{Neighbourhood Deformation Retract}\\
  OLS  &  Moindres carrés ordinaires, de l'anglais \textit{Ordinary Least Square}\\
  ZFC & Théorie des ensembles de Zermelo-Fraenkel avec l'axiome du choix\\
\end{twocolumnlist}
%% L'environnement <threecolumnlist> existe aussi pour trois colonnes.

%%------------------------------------------------- %
%%              pages vi                            %
%%------------------------------------------------- %

\chapter*{Remerciements}

...remerciements...

 %
 % Fin des pages liminaires.  À partir d'ici, les
 % premières pages des chapitres ne doivent pas
 % être numérotées
 %

\NoChapterPageNumber
\cleardoublepage
\pagenumbering{arabic}

%%%%%%%%%%%%%%%%%%%%%%%%%%%%%%%%%%%%%%%%%%%%%%%%%%%%%
%%                                                  %
%%   TEXTE DU MÉMOIRE :  introduction page 1,...    %
%%                                                  %
%%%%%%%%%%%%%%%%%%%%%%%%%%%%%%%%%%%%%%%%%%%%%%%%%%%%%

\chapter*{Introduction}


ce ci est une these 

% \include{0}
% \include{1}
% \include{2}
% \include{3}
% \include{4}
% \include{5}
% \include{6}



\endinput
%%
%% End of file `gabaritmem.tex'.
\end
